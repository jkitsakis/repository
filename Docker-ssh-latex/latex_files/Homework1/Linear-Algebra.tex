\documentclass[11pt]{article}

    \usepackage[breakable]{tcolorbox}
    \usepackage{parskip} % Stop auto-indenting (to mimic markdown behaviour)
    

    % Basic figure setup, for now with no caption control since it's done
    % automatically by Pandoc (which extracts ![](path) syntax from Markdown).
    \usepackage{graphicx}
    % Maintain compatibility with old templates. Remove in nbconvert 6.0
    \let\Oldincludegraphics\includegraphics
    % Ensure that by default, figures have no caption (until we provide a
    % proper Figure object with a Caption API and a way to capture that
    % in the conversion process - todo).
    \usepackage{caption}
    \DeclareCaptionFormat{nocaption}{}
    \captionsetup{format=nocaption,aboveskip=0pt,belowskip=0pt}

    \usepackage{float}
    \floatplacement{figure}{H} % forces figures to be placed at the correct location
    \usepackage{xcolor} % Allow colors to be defined
    \usepackage{enumerate} % Needed for markdown enumerations to work
    \usepackage{geometry} % Used to adjust the document margins
    \usepackage{amsmath} % Equations
    \usepackage{amssymb} % Equations
    \usepackage{textcomp} % defines textquotesingle
    % Hack from http://tex.stackexchange.com/a/47451/13684:
    \AtBeginDocument{%
        \def\PYZsq{\textquotesingle}% Upright quotes in Pygmentized code
    }
    \usepackage{upquote} % Upright quotes for verbatim code
    \usepackage{eurosym} % defines \euro

    \usepackage{iftex}
    \ifPDFTeX
        \usepackage[T1]{fontenc}
        \IfFileExists{alphabeta.sty}{
              \usepackage{alphabeta}
          }{
              \usepackage[mathletters]{ucs}
              \usepackage[utf8x]{inputenc}
          }
    \else
        \usepackage{fontspec}
        \usepackage{unicode-math}
    \fi

    \usepackage{fancyvrb} % verbatim replacement that allows latex
    \usepackage[Export]{adjustbox} % Used to constrain images to a maximum size
    \adjustboxset{max size={0.9\linewidth}{0.9\paperheight}}

    % The hyperref package gives us a pdf with properly built
    % internal navigation ('pdf bookmarks' for the table of contents,
    % internal cross-reference links, web links for URLs, etc.)
    \usepackage{hyperref}
    % The default LaTeX title has an obnoxious amount of whitespace. By default,
    % titling removes some of it. It also provides customization options.
    \usepackage{titling}
    \usepackage{longtable} % longtable support required by pandoc >1.10
    \usepackage{booktabs}  % table support for pandoc > 1.12.2
    \usepackage{array}     % table support for pandoc >= 2.11.3
    \usepackage{calc}      % table minipage width calculation for pandoc >= 2.11.1
    \usepackage[inline]{enumitem} % IRkernel/repr support (it uses the enumerate* environment)
    \usepackage[normalem]{ulem} % ulem is needed to support strikethroughs (\sout)
                                % normalem makes italics be italics, not underlines
    \usepackage{mathrsfs}
    

    
    % Colors for the hyperref package
    \definecolor{urlcolor}{rgb}{0,.145,.698}
    \definecolor{linkcolor}{rgb}{.71,0.21,0.01}
    \definecolor{citecolor}{rgb}{.12,.54,.11}

    % ANSI colors
    \definecolor{ansi-black}{HTML}{3E424D}
    \definecolor{ansi-black-intense}{HTML}{282C36}
    \definecolor{ansi-red}{HTML}{E75C58}
    \definecolor{ansi-red-intense}{HTML}{B22B31}
    \definecolor{ansi-green}{HTML}{00A250}
    \definecolor{ansi-green-intense}{HTML}{007427}
    \definecolor{ansi-yellow}{HTML}{DDB62B}
    \definecolor{ansi-yellow-intense}{HTML}{B27D12}
    \definecolor{ansi-blue}{HTML}{208FFB}
    \definecolor{ansi-blue-intense}{HTML}{0065CA}
    \definecolor{ansi-magenta}{HTML}{D160C4}
    \definecolor{ansi-magenta-intense}{HTML}{A03196}
    \definecolor{ansi-cyan}{HTML}{60C6C8}
    \definecolor{ansi-cyan-intense}{HTML}{258F8F}
    \definecolor{ansi-white}{HTML}{C5C1B4}
    \definecolor{ansi-white-intense}{HTML}{A1A6B2}
    \definecolor{ansi-default-inverse-fg}{HTML}{FFFFFF}
    \definecolor{ansi-default-inverse-bg}{HTML}{000000}

    % common color for the border for error outputs.
    \definecolor{outerrorbackground}{HTML}{FFDFDF}

    % commands and environments needed by pandoc snippets
    % extracted from the output of `pandoc -s`
    \providecommand{\tightlist}{%
      \setlength{\itemsep}{0pt}\setlength{\parskip}{0pt}}
    \DefineVerbatimEnvironment{Highlighting}{Verbatim}{commandchars=\\\{\}}
    % Add ',fontsize=\small' for more characters per line
    \newenvironment{Shaded}{}{}
    \newcommand{\KeywordTok}[1]{\textcolor[rgb]{0.00,0.44,0.13}{\textbf{{#1}}}}
    \newcommand{\DataTypeTok}[1]{\textcolor[rgb]{0.56,0.13,0.00}{{#1}}}
    \newcommand{\DecValTok}[1]{\textcolor[rgb]{0.25,0.63,0.44}{{#1}}}
    \newcommand{\BaseNTok}[1]{\textcolor[rgb]{0.25,0.63,0.44}{{#1}}}
    \newcommand{\FloatTok}[1]{\textcolor[rgb]{0.25,0.63,0.44}{{#1}}}
    \newcommand{\CharTok}[1]{\textcolor[rgb]{0.25,0.44,0.63}{{#1}}}
    \newcommand{\StringTok}[1]{\textcolor[rgb]{0.25,0.44,0.63}{{#1}}}
    \newcommand{\CommentTok}[1]{\textcolor[rgb]{0.38,0.63,0.69}{\textit{{#1}}}}
    \newcommand{\OtherTok}[1]{\textcolor[rgb]{0.00,0.44,0.13}{{#1}}}
    \newcommand{\AlertTok}[1]{\textcolor[rgb]{1.00,0.00,0.00}{\textbf{{#1}}}}
    \newcommand{\FunctionTok}[1]{\textcolor[rgb]{0.02,0.16,0.49}{{#1}}}
    \newcommand{\RegionMarkerTok}[1]{{#1}}
    \newcommand{\ErrorTok}[1]{\textcolor[rgb]{1.00,0.00,0.00}{\textbf{{#1}}}}
    \newcommand{\NormalTok}[1]{{#1}}
    
    % Additional commands for more recent versions of Pandoc
    \newcommand{\ConstantTok}[1]{\textcolor[rgb]{0.53,0.00,0.00}{{#1}}}
    \newcommand{\SpecialCharTok}[1]{\textcolor[rgb]{0.25,0.44,0.63}{{#1}}}
    \newcommand{\VerbatimStringTok}[1]{\textcolor[rgb]{0.25,0.44,0.63}{{#1}}}
    \newcommand{\SpecialStringTok}[1]{\textcolor[rgb]{0.73,0.40,0.53}{{#1}}}
    \newcommand{\ImportTok}[1]{{#1}}
    \newcommand{\DocumentationTok}[1]{\textcolor[rgb]{0.73,0.13,0.13}{\textit{{#1}}}}
    \newcommand{\AnnotationTok}[1]{\textcolor[rgb]{0.38,0.63,0.69}{\textbf{\textit{{#1}}}}}
    \newcommand{\CommentVarTok}[1]{\textcolor[rgb]{0.38,0.63,0.69}{\textbf{\textit{{#1}}}}}
    \newcommand{\VariableTok}[1]{\textcolor[rgb]{0.10,0.09,0.49}{{#1}}}
    \newcommand{\ControlFlowTok}[1]{\textcolor[rgb]{0.00,0.44,0.13}{\textbf{{#1}}}}
    \newcommand{\OperatorTok}[1]{\textcolor[rgb]{0.40,0.40,0.40}{{#1}}}
    \newcommand{\BuiltInTok}[1]{{#1}}
    \newcommand{\ExtensionTok}[1]{{#1}}
    \newcommand{\PreprocessorTok}[1]{\textcolor[rgb]{0.74,0.48,0.00}{{#1}}}
    \newcommand{\AttributeTok}[1]{\textcolor[rgb]{0.49,0.56,0.16}{{#1}}}
    \newcommand{\InformationTok}[1]{\textcolor[rgb]{0.38,0.63,0.69}{\textbf{\textit{{#1}}}}}
    \newcommand{\WarningTok}[1]{\textcolor[rgb]{0.38,0.63,0.69}{\textbf{\textit{{#1}}}}}
    
    
    % Define a nice break command that doesn't care if a line doesn't already
    % exist.
    \def\br{\hspace*{\fill} \\* }
    % Math Jax compatibility definitions
    \def\gt{>}
    \def\lt{<}
    \let\Oldtex\TeX
    \let\Oldlatex\LaTeX
    \renewcommand{\TeX}{\textrm{\Oldtex}}
    \renewcommand{\LaTeX}{\textrm{\Oldlatex}}
    % Document parameters
    % Document title
    \title{Linear Algebra}
    
    
    
    
    
% Pygments definitions
\makeatletter
\def\PY@reset{\let\PY@it=\relax \let\PY@bf=\relax%
    \let\PY@ul=\relax \let\PY@tc=\relax%
    \let\PY@bc=\relax \let\PY@ff=\relax}
\def\PY@tok#1{\csname PY@tok@#1\endcsname}
\def\PY@toks#1+{\ifx\relax#1\empty\else%
    \PY@tok{#1}\expandafter\PY@toks\fi}
\def\PY@do#1{\PY@bc{\PY@tc{\PY@ul{%
    \PY@it{\PY@bf{\PY@ff{#1}}}}}}}
\def\PY#1#2{\PY@reset\PY@toks#1+\relax+\PY@do{#2}}

\@namedef{PY@tok@w}{\def\PY@tc##1{\textcolor[rgb]{0.73,0.73,0.73}{##1}}}
\@namedef{PY@tok@c}{\let\PY@it=\textit\def\PY@tc##1{\textcolor[rgb]{0.24,0.48,0.48}{##1}}}
\@namedef{PY@tok@cp}{\def\PY@tc##1{\textcolor[rgb]{0.61,0.40,0.00}{##1}}}
\@namedef{PY@tok@k}{\let\PY@bf=\textbf\def\PY@tc##1{\textcolor[rgb]{0.00,0.50,0.00}{##1}}}
\@namedef{PY@tok@kp}{\def\PY@tc##1{\textcolor[rgb]{0.00,0.50,0.00}{##1}}}
\@namedef{PY@tok@kt}{\def\PY@tc##1{\textcolor[rgb]{0.69,0.00,0.25}{##1}}}
\@namedef{PY@tok@o}{\def\PY@tc##1{\textcolor[rgb]{0.40,0.40,0.40}{##1}}}
\@namedef{PY@tok@ow}{\let\PY@bf=\textbf\def\PY@tc##1{\textcolor[rgb]{0.67,0.13,1.00}{##1}}}
\@namedef{PY@tok@nb}{\def\PY@tc##1{\textcolor[rgb]{0.00,0.50,0.00}{##1}}}
\@namedef{PY@tok@nf}{\def\PY@tc##1{\textcolor[rgb]{0.00,0.00,1.00}{##1}}}
\@namedef{PY@tok@nc}{\let\PY@bf=\textbf\def\PY@tc##1{\textcolor[rgb]{0.00,0.00,1.00}{##1}}}
\@namedef{PY@tok@nn}{\let\PY@bf=\textbf\def\PY@tc##1{\textcolor[rgb]{0.00,0.00,1.00}{##1}}}
\@namedef{PY@tok@ne}{\let\PY@bf=\textbf\def\PY@tc##1{\textcolor[rgb]{0.80,0.25,0.22}{##1}}}
\@namedef{PY@tok@nv}{\def\PY@tc##1{\textcolor[rgb]{0.10,0.09,0.49}{##1}}}
\@namedef{PY@tok@no}{\def\PY@tc##1{\textcolor[rgb]{0.53,0.00,0.00}{##1}}}
\@namedef{PY@tok@nl}{\def\PY@tc##1{\textcolor[rgb]{0.46,0.46,0.00}{##1}}}
\@namedef{PY@tok@ni}{\let\PY@bf=\textbf\def\PY@tc##1{\textcolor[rgb]{0.44,0.44,0.44}{##1}}}
\@namedef{PY@tok@na}{\def\PY@tc##1{\textcolor[rgb]{0.41,0.47,0.13}{##1}}}
\@namedef{PY@tok@nt}{\let\PY@bf=\textbf\def\PY@tc##1{\textcolor[rgb]{0.00,0.50,0.00}{##1}}}
\@namedef{PY@tok@nd}{\def\PY@tc##1{\textcolor[rgb]{0.67,0.13,1.00}{##1}}}
\@namedef{PY@tok@s}{\def\PY@tc##1{\textcolor[rgb]{0.73,0.13,0.13}{##1}}}
\@namedef{PY@tok@sd}{\let\PY@it=\textit\def\PY@tc##1{\textcolor[rgb]{0.73,0.13,0.13}{##1}}}
\@namedef{PY@tok@si}{\let\PY@bf=\textbf\def\PY@tc##1{\textcolor[rgb]{0.64,0.35,0.47}{##1}}}
\@namedef{PY@tok@se}{\let\PY@bf=\textbf\def\PY@tc##1{\textcolor[rgb]{0.67,0.36,0.12}{##1}}}
\@namedef{PY@tok@sr}{\def\PY@tc##1{\textcolor[rgb]{0.64,0.35,0.47}{##1}}}
\@namedef{PY@tok@ss}{\def\PY@tc##1{\textcolor[rgb]{0.10,0.09,0.49}{##1}}}
\@namedef{PY@tok@sx}{\def\PY@tc##1{\textcolor[rgb]{0.00,0.50,0.00}{##1}}}
\@namedef{PY@tok@m}{\def\PY@tc##1{\textcolor[rgb]{0.40,0.40,0.40}{##1}}}
\@namedef{PY@tok@gh}{\let\PY@bf=\textbf\def\PY@tc##1{\textcolor[rgb]{0.00,0.00,0.50}{##1}}}
\@namedef{PY@tok@gu}{\let\PY@bf=\textbf\def\PY@tc##1{\textcolor[rgb]{0.50,0.00,0.50}{##1}}}
\@namedef{PY@tok@gd}{\def\PY@tc##1{\textcolor[rgb]{0.63,0.00,0.00}{##1}}}
\@namedef{PY@tok@gi}{\def\PY@tc##1{\textcolor[rgb]{0.00,0.52,0.00}{##1}}}
\@namedef{PY@tok@gr}{\def\PY@tc##1{\textcolor[rgb]{0.89,0.00,0.00}{##1}}}
\@namedef{PY@tok@ge}{\let\PY@it=\textit}
\@namedef{PY@tok@gs}{\let\PY@bf=\textbf}
\@namedef{PY@tok@gp}{\let\PY@bf=\textbf\def\PY@tc##1{\textcolor[rgb]{0.00,0.00,0.50}{##1}}}
\@namedef{PY@tok@go}{\def\PY@tc##1{\textcolor[rgb]{0.44,0.44,0.44}{##1}}}
\@namedef{PY@tok@gt}{\def\PY@tc##1{\textcolor[rgb]{0.00,0.27,0.87}{##1}}}
\@namedef{PY@tok@err}{\def\PY@bc##1{{\setlength{\fboxsep}{\string -\fboxrule}\fcolorbox[rgb]{1.00,0.00,0.00}{1,1,1}{\strut ##1}}}}
\@namedef{PY@tok@kc}{\let\PY@bf=\textbf\def\PY@tc##1{\textcolor[rgb]{0.00,0.50,0.00}{##1}}}
\@namedef{PY@tok@kd}{\let\PY@bf=\textbf\def\PY@tc##1{\textcolor[rgb]{0.00,0.50,0.00}{##1}}}
\@namedef{PY@tok@kn}{\let\PY@bf=\textbf\def\PY@tc##1{\textcolor[rgb]{0.00,0.50,0.00}{##1}}}
\@namedef{PY@tok@kr}{\let\PY@bf=\textbf\def\PY@tc##1{\textcolor[rgb]{0.00,0.50,0.00}{##1}}}
\@namedef{PY@tok@bp}{\def\PY@tc##1{\textcolor[rgb]{0.00,0.50,0.00}{##1}}}
\@namedef{PY@tok@fm}{\def\PY@tc##1{\textcolor[rgb]{0.00,0.00,1.00}{##1}}}
\@namedef{PY@tok@vc}{\def\PY@tc##1{\textcolor[rgb]{0.10,0.09,0.49}{##1}}}
\@namedef{PY@tok@vg}{\def\PY@tc##1{\textcolor[rgb]{0.10,0.09,0.49}{##1}}}
\@namedef{PY@tok@vi}{\def\PY@tc##1{\textcolor[rgb]{0.10,0.09,0.49}{##1}}}
\@namedef{PY@tok@vm}{\def\PY@tc##1{\textcolor[rgb]{0.10,0.09,0.49}{##1}}}
\@namedef{PY@tok@sa}{\def\PY@tc##1{\textcolor[rgb]{0.73,0.13,0.13}{##1}}}
\@namedef{PY@tok@sb}{\def\PY@tc##1{\textcolor[rgb]{0.73,0.13,0.13}{##1}}}
\@namedef{PY@tok@sc}{\def\PY@tc##1{\textcolor[rgb]{0.73,0.13,0.13}{##1}}}
\@namedef{PY@tok@dl}{\def\PY@tc##1{\textcolor[rgb]{0.73,0.13,0.13}{##1}}}
\@namedef{PY@tok@s2}{\def\PY@tc##1{\textcolor[rgb]{0.73,0.13,0.13}{##1}}}
\@namedef{PY@tok@sh}{\def\PY@tc##1{\textcolor[rgb]{0.73,0.13,0.13}{##1}}}
\@namedef{PY@tok@s1}{\def\PY@tc##1{\textcolor[rgb]{0.73,0.13,0.13}{##1}}}
\@namedef{PY@tok@mb}{\def\PY@tc##1{\textcolor[rgb]{0.40,0.40,0.40}{##1}}}
\@namedef{PY@tok@mf}{\def\PY@tc##1{\textcolor[rgb]{0.40,0.40,0.40}{##1}}}
\@namedef{PY@tok@mh}{\def\PY@tc##1{\textcolor[rgb]{0.40,0.40,0.40}{##1}}}
\@namedef{PY@tok@mi}{\def\PY@tc##1{\textcolor[rgb]{0.40,0.40,0.40}{##1}}}
\@namedef{PY@tok@il}{\def\PY@tc##1{\textcolor[rgb]{0.40,0.40,0.40}{##1}}}
\@namedef{PY@tok@mo}{\def\PY@tc##1{\textcolor[rgb]{0.40,0.40,0.40}{##1}}}
\@namedef{PY@tok@ch}{\let\PY@it=\textit\def\PY@tc##1{\textcolor[rgb]{0.24,0.48,0.48}{##1}}}
\@namedef{PY@tok@cm}{\let\PY@it=\textit\def\PY@tc##1{\textcolor[rgb]{0.24,0.48,0.48}{##1}}}
\@namedef{PY@tok@cpf}{\let\PY@it=\textit\def\PY@tc##1{\textcolor[rgb]{0.24,0.48,0.48}{##1}}}
\@namedef{PY@tok@c1}{\let\PY@it=\textit\def\PY@tc##1{\textcolor[rgb]{0.24,0.48,0.48}{##1}}}
\@namedef{PY@tok@cs}{\let\PY@it=\textit\def\PY@tc##1{\textcolor[rgb]{0.24,0.48,0.48}{##1}}}

\def\PYZbs{\char`\\}
\def\PYZus{\char`\_}
\def\PYZob{\char`\{}
\def\PYZcb{\char`\}}
\def\PYZca{\char`\^}
\def\PYZam{\char`\&}
\def\PYZlt{\char`\<}
\def\PYZgt{\char`\>}
\def\PYZsh{\char`\#}
\def\PYZpc{\char`\%}
\def\PYZdl{\char`$ }
\def\PYZhy{\char`\-}
\def\PYZsq{\char`\'}
\def\PYZdq{\char`\"}
\def\PYZti{\char`\~}
% for compatibility with earlier versions
\def\PYZat{@}
\def\PYZlb{[}
\def\PYZrb{]}
\makeatother


    % For linebreaks inside Verbatim environment from package fancyvrb. 
    \makeatletter
        \newbox\Wrappedcontinuationbox 
        \newbox\Wrappedvisiblespacebox 
        \newcommand*\Wrappedvisiblespace {\textcolor{red}{\textvisiblespace}} 
        \newcommand*\Wrappedcontinuationsymbol {\textcolor{red}{\llap{\tiny$\m@th\hookrightarrow$}}} 
        \newcommand*\Wrappedcontinuationindent {3ex } 
        \newcommand*\Wrappedafterbreak {\kern\Wrappedcontinuationindent\copy\Wrappedcontinuationbox} 
        % Take advantage of the already applied Pygments mark-up to insert 
        % potential linebreaks for TeX processing. 
        %        {, <, #, %, $, ' and ": go to next line. 
        %        _, }, ^, &, >, - and ~: stay at end of broken line. 
        % Use of \textquotesingle for straight quote. 
        \newcommand*\Wrappedbreaksatspecials {% 
            \def\PYGZus{\discretionary{\char`\_}{\Wrappedafterbreak}{\char`\_}}% 
            \def\PYGZob{\discretionary{}{\Wrappedafterbreak\char`\{}{\char`\{}}% 
            \def\PYGZcb{\discretionary{\char`\}}{\Wrappedafterbreak}{\char`\}}}% 
            \def\PYGZca{\discretionary{\char`\^}{\Wrappedafterbreak}{\char`\^}}% 
            \def\PYGZam{\discretionary{\char`\&}{\Wrappedafterbreak}{\char`\&}}% 
            \def\PYGZlt{\discretionary{}{\Wrappedafterbreak\char`\<}{\char`\<}}% 
            \def\PYGZgt{\discretionary{\char`\>}{\Wrappedafterbreak}{\char`\>}}% 
            \def\PYGZsh{\discretionary{}{\Wrappedafterbreak\char`\#}{\char`\#}}% 
            \def\PYGZpc{\discretionary{}{\Wrappedafterbreak\char`\%}{\char`\%}}% 
            \def\PYGZdl{\discretionary{}{\Wrappedafterbreak\char`$ }{\char`$ }}% 
            \def\PYGZhy{\discretionary{\char`\-}{\Wrappedafterbreak}{\char`\-}}% 
            \def\PYGZsq{\discretionary{}{\Wrappedafterbreak\textquotesingle}{\textquotesingle}}% 
            \def\PYGZdq{\discretionary{}{\Wrappedafterbreak\char`\"}{\char`\"}}% 
            \def\PYGZti{\discretionary{\char`\~}{\Wrappedafterbreak}{\char`\~}}% 
        } 
        % Some characters . , ; ? ! / are not pygmentized. 
        % This macro makes them "active" and they will insert potential linebreaks 
        \newcommand*\Wrappedbreaksatpunct {% 
            \lccode`\~`\.\lowercase{\def~}{\discretionary{\hbox{\char`\.}}{\Wrappedafterbreak}{\hbox{\char`\.}}}% 
            \lccode`\~`\,\lowercase{\def~}{\discretionary{\hbox{\char`\,}}{\Wrappedafterbreak}{\hbox{\char`\,}}}% 
            \lccode`\~`\;\lowercase{\def~}{\discretionary{\hbox{\char`\;}}{\Wrappedafterbreak}{\hbox{\char`\;}}}% 
            \lccode`\~`\:\lowercase{\def~}{\discretionary{\hbox{\char`\:}}{\Wrappedafterbreak}{\hbox{\char`\:}}}% 
            \lccode`\~`\?\lowercase{\def~}{\discretionary{\hbox{\char`\?}}{\Wrappedafterbreak}{\hbox{\char`\?}}}% 
            \lccode`\~`\!\lowercase{\def~}{\discretionary{\hbox{\char`\!}}{\Wrappedafterbreak}{\hbox{\char`\!}}}% 
            \lccode`\~`\/\lowercase{\def~}{\discretionary{\hbox{\char`\/}}{\Wrappedafterbreak}{\hbox{\char`\/}}}% 
            \catcode`\.\active
            \catcode`\,\active 
            \catcode`\;\active
            \catcode`\:\active
            \catcode`\?\active
            \catcode`\!\active
            \catcode`\/\active 
            \lccode`\~`\~ 	
        }
    \makeatother

    \let\OriginalVerbatim=\Verbatim
    \makeatletter
    \renewcommand{\Verbatim}[1][1]{%
        %\parskip\z@skip
        \sbox\Wrappedcontinuationbox {\Wrappedcontinuationsymbol}%
        \sbox\Wrappedvisiblespacebox {\FV@SetupFont\Wrappedvisiblespace}%
        \def\FancyVerbFormatLine ##1{\hsize\linewidth
            \vtop{\raggedright\hyphenpenalty\z@\exhyphenpenalty\z@
                \doublehyphendemerits\z@\finalhyphendemerits\z@
                \strut ##1\strut}%
        }%
        % If the linebreak is at a space, the latter will be displayed as visible
        % space at end of first line, and a continuation symbol starts next line.
        % Stretch/shrink are however usually zero for typewriter font.
        \def\FV@Space {%
            \nobreak\hskip\z@ plus\fontdimen3\font minus\fontdimen4\font
            \discretionary{\copy\Wrappedvisiblespacebox}{\Wrappedafterbreak}
            {\kern\fontdimen2\font}%
        }%
        
        % Allow breaks at special characters using \PYG... macros.
        \Wrappedbreaksatspecials
        % Breaks at punctuation characters . , ; ? ! and / need catcode=\active 	
        \OriginalVerbatim[#1,codes*=\Wrappedbreaksatpunct]%
    }
    \makeatother

    % Exact colors from NB
    \definecolor{incolor}{HTML}{303F9F}
    \definecolor{outcolor}{HTML}{D84315}
    \definecolor{cellborder}{HTML}{CFCFCF}
    \definecolor{cellbackground}{HTML}{F7F7F7}
    
    % prompt
    \makeatletter
    \newcommand{\boxspacing}{\kern\kvtcb@left@rule\kern\kvtcb@boxsep}
    \makeatother
    \newcommand{\prompt}[4]{
        {\ttfamily\llap{{\color{#2}[#3]:\hspace{3pt}#4}}\vspace{-\baselineskip}}
    }
    

    
    % Prevent overflowing lines due to hard-to-break entities
    \sloppy 
    % Setup hyperref package
    \hypersetup{
      breaklinks=true,  % so long urls are correctly broken across lines
      colorlinks=true,
      urlcolor=urlcolor,
      linkcolor=linkcolor,
      citecolor=citecolor,
      }
    % Slightly bigger margins than the latex defaults
    
    \geometry{verbose,tmargin=1in,bmargin=1in,lmargin=1in,rmargin=1in}
    
    

\begin{document}
    
    \maketitle
    
    

    
    \begin{tcolorbox}[breakable, size=fbox, boxrule=1pt, pad at break*=1mm,colback=cellbackground, colframe=cellborder]
\prompt{In}{incolor}{2}{\boxspacing}
\begin{Verbatim}[commandchars=\\\{\}]
\PY{o}{\PYZpc{}}\PY{k}{display} latex
\end{Verbatim}
\end{tcolorbox}

    \hypertarget{what-is-gaussian-elimination}{%
\subsubsection{What is Gaussian Elimination
?}\label{what-is-gaussian-elimination}}

    \textbf{Gaussian Elimination:}

Gaussian Elimination is a method used to solve systems of linear
equations by transforming the augmented matrix of the system to its
row-echelon form (or reduced row-echelon form). The process involves a
sequence of elementary row operations to simplify the matrix.

Here's an example in Markdown with LaTeX math mode:

Consider the system of linear equations:

\begin{align*}
2x + 3y - z &= 1 \\
4x - y + z &= 5 \\
-2x + 2y + 3z &= -4
\end{align*}

We can represent this system in augmented matrix form as:

\(\begin{bmatrix} 2 & 3 & -1 & | & 1 \\ 4 & -1 & 1 & | & 5 \\ -2 & 2 & 3 & | & -4 \end{bmatrix}\)

Now, let's perform Gaussian Elimination to transform this matrix into
row-echelon form:

\begin{enumerate}
\def\labelenumi{\arabic{enumi}.}
\tightlist
\item
  \textbf{Row 2 - 2 * Row 1:}
\end{enumerate}

\(\begin{bmatrix} 2 & 3 & -1 & | & 1 \\ 0 & -7 & 3 & | & 3 \\ -2 & 2 & 3 & | & -4 \end{bmatrix}\)

\begin{enumerate}
\def\labelenumi{\arabic{enumi}.}
\setcounter{enumi}{1}
\tightlist
\item
  \textbf{Row 3 + Row 1:}
\end{enumerate}

\(\begin{bmatrix} 2 & 3 & -1 & | & 1 \\ 0 & -7 & 3 & | & 3 \\ 0 & 5 & 2 & | & -3 \end{bmatrix}\)

\begin{enumerate}
\def\labelenumi{\arabic{enumi}.}
\setcounter{enumi}{2}
\tightlist
\item
  \textbf{Row 3 + (5/7) * Row 2:}
\end{enumerate}

\(\begin{bmatrix} 2 & 3 & -1 & | & 1 \\ 0 & -7 & 3 & | & 3 \\ 0 & 0 & \frac{11}{7} & | & -\frac{24}{7} \end{bmatrix}\)

Now, the system can be easily solved back-substituting:


$\frac{11}{7}z$ = -$\frac{24}{7}$ $\implies z = -2$ \\
-7y + 3z = 3 $\implies y = 1$ \\
2x + 3y - z = 1 $\implies x = 2$


Therefore, the solution to the system is (x = 2), (y = 1), and (z = -2).

    

    \hypertarget{what-is-matrix-inversion}{%
\subsubsection{What is Matrix Inversion
?}\label{what-is-matrix-inversion}}

    \textbf{Matrix Inversion:}

Matrix inversion is the process of finding the inverse of a square
matrix. The inverse of a matrix \(A\), denoted as \(A^{-1}\), is a
matrix such that when \(A\) is multiplied by its inverse, the result is
the identity matrix \(I\).

Here's an example in Markdown with LaTeX math mode:

Consider a 2x2 matrix:

A = \(\begin{bmatrix} a & b \\ c & d \end{bmatrix}\)

The inverse of (A), if it exists, is given by:

$A^-1 = \frac{1}{ad - bc}$
$\begin{bmatrix} d & -b \\ -c & a \end{bmatrix}$

Let's say we have the following matrix:

B = $\begin{bmatrix} 2 & 3 \\ 1 & 4 \end{bmatrix}$

We want to find the inverse $B^-1$. First, we check if the
determinant (ad - bc) is non-zero.

$\text{det}(B) = (2 \cdot 4) - (3 \cdot 1) = 5 \neq 0$

So, the inverse exists.

Now, we can find \(B^{-1}\):

\(B^{-1}\) = \(\frac{1}{5}\)
\(\begin{bmatrix} 4 & -3 \\ -1 & 2 \end{bmatrix}\)

This means that:

$ B \cdot B^-1 = \begin{bmatrix} 2 & 3 \\ 1 & 4 \end{bmatrix}
\cdot \frac{1}{5} \begin{bmatrix} 4 & -3 \\ -1 & 2 \end{bmatrix} =
\begin{bmatrix} 1 & 0 \\ 0 & 1 \end{bmatrix} = I $

So, the product of the matrix and its inverse is the identity matrix,
confirming that \(B^{-1}\)) is indeed the inverse of (B).

    \begin{tcolorbox}[breakable, size=fbox, boxrule=1pt, pad at break*=1mm,colback=cellbackground, colframe=cellborder]
\prompt{In}{incolor}{ }{\boxspacing}
\begin{Verbatim}[commandchars=\\\{\}]

\end{Verbatim}
\end{tcolorbox}

    \hypertarget{explain-adjugate-matrix-of-2x2-and-3x3-matrices}{%
\subsubsection{Explain Adjugate matrix of 2x2 and 3x3
matrices}\label{explain-adjugate-matrix-of-2x2-and-3x3-matrices}}

    \textbf{Adjugate Matrix of 2x2 and 3x3 Matrices:}

The adjugate matrix, also known as the adjoint matrix, of a square
matrix $  A $ is denoted as $  \text{adj}(A) $ or $  A^*$ . It is used in
the formula for finding the inverse of a matrix.

For a 2x2 matrix:

A = $\begin{bmatrix} a & b \\ c & d \end{bmatrix}$

The adjugate matrix is:

$  \text{adj}(A) =\begin{bmatrix} d & -b \\ -c & a \end{bmatrix}$ 

For a 3x3 matrix:

B =$\begin{bmatrix} a & b & c \\ d & e & f \\ g & h & i \end{bmatrix}$

The adjugate matrix is formed by the cofactors of the elements of the
original matrix, transposed:

$ \text{adj}(B) =\begin{bmatrix} C_{11} & C_{21} & C_{31} \\ C_{12} & C_{22} & C_{32} \\ C_{13} & C_{23} & C_{33} \end{bmatrix}^T $

Where $  C\_\{ij\} $  is the cofactor of the element at the $  i-th $ 
row and $  j-th$  column , given by:

$ C_ij = (-1)^i+j \cdot \text{det}(M_ij) $

Here, $  M\_\{ij\} $  is the matrix obtained by removing the $  i-th $ 
row and $  j-th $  column from $  B $ .

Now, let's express these concepts in Markdown with LaTeX math mode:

\textbf{For a 2x2 matrix:}

Let $  A =\begin{bmatrix} a & b \\ c & d \end{bmatrix}$ , then the adjugate matrix is:

$  \text{adj}(A) =\begin{bmatrix} d & -b \\ -c & a \end{bmatrix}$

$ \textbf{For a 3x3 matrix:} $

Let B =
$\begin{bmatrix} a & b & c \\ d & e & f \\ g & h & i \end{bmatrix}$,
then the adjugate matrix is:

$ \text{adj}(B) = \begin{bmatrix} C_{11} & C_{21} & C_{31} \\ C_{12} & C_{22} & C_{32} \\ C_{13} & C_{23} & C_{33} \end{bmatrix}^T $

where $  C_ij = (-1)^i+j \cdot \text{det}(M_ij)$ , and
$  M_ij$  is the matrix obtained by removing the $  i-th $ row and$  j-th $  column from $  B $ .

    \begin{tcolorbox}[breakable, size=fbox, boxrule=1pt, pad at break*=1mm,colback=cellbackground, colframe=cellborder]
\prompt{In}{incolor}{ }{\boxspacing}
\begin{Verbatim}[commandchars=\\\{\}]

\end{Verbatim}
\end{tcolorbox}

    \hypertarget{what-is-row-echelon-form}{%
\subsubsection{What is row echelon form
?}\label{what-is-row-echelon-form}}

    \textbf{Row Echelon Form:}

Row Echelon Form (REF) is a special form that a matrix can be
transformed into using a sequence of elementary row operations. A matrix
is in row echelon form if it satisfies the following conditions:

\begin{enumerate}
\def\labelenumi{\arabic{enumi}.}
\tightlist
\item
  All zero rows, if any, are at the bottom.
\item
  The leading entry (the leftmost nonzero entry) of each nonzero row
  occurs to the right of the leading entry of the previous row.
\item
  The leading entry in each nonzero row is 1.
\item
  The leading 1 in each nonzero row is the only nonzero entry in its
  column.
\end{enumerate}

Here's an example in Markdown with LaTeX math mode:

Consider the matrix:

A =
\(\begin{bmatrix} 2 & 3 & -1 & 5 \\ 0 & 1 & 2 & 4 \\ 0 & 0 & 1 & 2 \end{bmatrix}\)

Now, let's perform elementary row operations to bring it to row echelon
form:

\begin{enumerate}
\def\labelenumi{\arabic{enumi}.}
\tightlist
\item
  \textbf{Row 1 divided by 2:}
\end{enumerate}

\(\begin{bmatrix} 1 & \frac{3}{2} & -\frac{1}{2} & \frac{5}{2} \\ 0 & 1 & 2 & 4 \\ 0 & 0 & 1 & 2 \end{bmatrix}\)

\begin{enumerate}
\def\labelenumi{\arabic{enumi}.}
\setcounter{enumi}{1}
\tightlist
\item
  \textbf{Row 1 - (3/2) * Row 2:}
\end{enumerate}

\(\begin{bmatrix} 1 & 0 & -4 & -3 \\ 0 & 1 & 2 & 4 \\ 0 & 0 & 1 & 2 \end{bmatrix}\)

This is the row echelon form of the original matrix (A). It satisfies
the conditions mentioned earlier. The leading entries are 1, and below
and above each leading 1, all entries are zeros.

    \begin{tcolorbox}[breakable, size=fbox, boxrule=1pt, pad at break*=1mm,colback=cellbackground, colframe=cellborder]
\prompt{In}{incolor}{ }{\boxspacing}
\begin{Verbatim}[commandchars=\\\{\}]

\end{Verbatim}
\end{tcolorbox}

    \hypertarget{augmented-matrix.}{%
\subsubsection{Augmented Matrix.}\label{augmented-matrix.}}

    \textbf{Augmented Matrix:}

An augmented matrix is a way to represent a system of linear equations
in matrix form. It combines the coefficient matrix and the column vector
of constants into a single matrix. For a system of (m) equations with
(n) variables, the augmented matrix is of the form ({[}A \textbar{}
B{]}), where (A) is the coefficient matrix and (B) is the column vector
of constants.

Here's an example in Markdown with LaTeX math mode:

Consider the following system of linear equations:

2x + 3y - z \&= 5 \textbackslash{} 4x - y + 2z \&= 8 \textbackslash{}
-2x + 2y + 3z \&= 1

The augmented matrix for this system is:

$  {[}A \textbar{} B{]} = \left[ \begin{array}{ccc|c}
2 & 3 & -1 & 5 \\
4 & -1 & 2 & 8 \\
-2 & 2 & 3 & 1
\end{array} \right]$ 

This augmented matrix contains the coefficients of the variables and the
constants from the system of equations. The vertical bar separates the
coefficient matrix ((A)) on the left from the column vector of constants
((B)) on the right.

    \begin{tcolorbox}[breakable, size=fbox, boxrule=1pt, pad at break*=1mm,colback=cellbackground, colframe=cellborder]
\prompt{In}{incolor}{ }{\boxspacing}
\begin{Verbatim}[commandchars=\\\{\}]

\end{Verbatim}
\end{tcolorbox}

    \hypertarget{solve-a-linear-system-with-augmented-matrix}{%
\subsubsection{Solve a Linear system with Augmented
Matrix}\label{solve-a-linear-system-with-augmented-matrix}}

    Certainly! Let's solve a linear system using the augmented matrix and
Gaussian elimination. Consider the following system:


2x + 3y - z = 5 \\
4x - y + 2z = 8 \\
-2x + 2y + 3z = 1


We can represent this system as an augmented matrix:

$ [A | B] = \left[ \begin{array}{ccc|c} 2 & 3 & -1 & 5 \\ 4 & -1 & 2 & 8 \\ -2 & 2 & 3 & 1 \end{array} \right]$ 

Now, let's perform Gaussian elimination to get the matrix into
row-echelon form:

\begin{enumerate}
\def\labelenumi{\arabic{enumi}.}
\tightlist
\item
  \textbf{Row 2 - 2 * Row 1:}
\end{enumerate}

$  \left[ \begin{array}{ccc|c}
2 & 3 & -1 & 5 \\
0 & -7 & 3 & -2 \\
-2 & 2 & 3 & 1
\end{array} \right]$ 

\begin{enumerate}
\def\labelenumi{\arabic{enumi}.}
\setcounter{enumi}{1}
\tightlist
\item
  \textbf{Row 3 + Row 1:}
\end{enumerate}

$  \left[ \begin{array}{ccc|c}
2 & 3 & -1 & 5 \\
0 & -7 & 3 & -2 \\
0 & 5 & 2 & 6
\end{array} \right]$ 

\begin{enumerate}
\def\labelenumi{\arabic{enumi}.}
\setcounter{enumi}{2}
\tightlist
\item
  \textbf{Row 3 + (5/7) * Row 2:}
\end{enumerate}

$  \left[ \begin{array}{ccc|c}
2 & 3 & -1 & 5 \\
0 & -7 & 3 & -2 \\
0 & 0 & \frac{11}{7} & \frac{28}{7}
\end{array} \right]$ 

Now, we can solve back-substituting:

z = 2

-7y + 3z = -2 $ \implies$  y = 1

2x + 3y - z = 5 $ \implies$  x = 2

So, the solution to the system is (x = 2), (y = 1), and (z = 2).

    \begin{tcolorbox}[breakable, size=fbox, boxrule=1pt, pad at break*=1mm,colback=cellbackground, colframe=cellborder]
\prompt{In}{incolor}{ }{\boxspacing}
\begin{Verbatim}[commandchars=\\\{\}]

\end{Verbatim}
\end{tcolorbox}

    \hypertarget{linear-independence}{%
\subsubsection{Linear Independence}\label{linear-independence}}

    \textbf{Linear Independence in Vectors:}

In linear algebra, a set of vectors is said to be linearly independent
if no vector in the set can be represented as a linear combination of
the others. In other words, no vector in the set is redundant; each
vector contributes uniquely to the span of the set.

For a set of vectors $  \mathbf{v}\_1, \mathbf{v}\_2, \ldots,\mathbf{v}\_n$  , 
these vectors are linearly independent if the equation

$ c_1 \mathbf{v}_1 + c_2 \mathbf{v}_2 + \ldots + c_n \mathbf{v}_n= \mathbf{0}$ 

has only the trivial solution $  c\_1 = c\_2 = \ldots = c\_n = 0$  .

\textbf{Example in $  \mathbb{R}^3$  :} Suppose we have the vectors:

$  \mathbf{v}_1 =\begin{bmatrix} 1 \\ 2 \\ -1 \end{bmatrix}, \quad \mathbf{v}\_2 =\begin{bmatrix} 0 \\ -1 \\ 2 \end{bmatrix}, \quad \mathbf{v}_3 =\begin{bmatrix} 3 \\ 0 \\ 1 \end{bmatrix}$ 

We want to determine if these vectors are linearly independent.

To check for linear independence, we can set up the following equation:

$  c\_1 \mathbf{v}\_1 + c\_2 \mathbf{v}\_2 + c\_3 \mathbf{v}\_3 =
\mathbf{0} $ 

This gives the system of equations:

$
c_1 + 0 + 3c_3 = 0 \\
2c_1 - c_2 + 0 = 0 \\
-c_1 + 2c_2 + c_3 = 0
$

We can represent this system in augmented matrix form and use Gaussian
elimination to solve for the coefficients $  c\_1, c\_2, c\_3$  :

$  {[}A \textbar{} B{]} = \left[ \begin{array}{ccc|c}
1 & 0 & 3 & 0 \\
2 & -1 & 0 & 0 \\
-1 & 2 & 1 & 0
\end{array} \right]$ 

Now, let's perform Gaussian elimination:

\begin{enumerate}
\def\labelenumi{\arabic{enumi}.}
\item
  \textbf{Row 2 - 2 * Row 1:} $  \left[ \begin{array}{ccc|c}
  1 & 0 & 3 & 0 \\
  0 & -1 & -6 & 0 \\
  -1 & 2 & 1 & 0
  \end{array} \right]$ 
\item
  \textbf{Row 3 + Row 1:} $  \left[ \begin{array}{ccc|c}
  1 & 0 & 3 & 0 \\
  0 & -1 & -6 & 0 \\
  0 & 2 & 4 & 0
  \end{array} \right]$ 
\item
  \textbf{Row 3 + 2 * Row 2:} $  \left[ \begin{array}{ccc|c}
  1 & 0 & 3 & 0 \\
  0 & -1 & -6 & 0 \\
  0 & 0 & -8 & 0
  \end{array} \right]$ 
\end{enumerate}

Now, the system is in row-echelon form. The third equation $  0x + 0y -
8z = 0$  implies $  z = 0$  . Substituting this back, the second
equation becomes $  -y + 6z = 0$  , which implies $  y = 0$  . Finally,
the first equation $  x + 3z = 0$  implies $  x = 0$  .

Since the only solution is the trivial solution $  c\_1 = c\_2 = c\_3 =
0$  , the vectors are linearly independent.

    \begin{tcolorbox}[breakable, size=fbox, boxrule=1pt, pad at break*=1mm,colback=cellbackground, colframe=cellborder]
\prompt{In}{incolor}{ }{\boxspacing}
\begin{Verbatim}[commandchars=\\\{\}]

\end{Verbatim}
\end{tcolorbox}

    Basis and Rank in Vectors. Example in $  R^3 $ ,

    \textbf{Basis and Rank in $  \mathbb{R}^3)$:}

A basis for a vector space is a set of vectors that spans the space and
is linearly independent. The rank of a set of vectors is the maximum
number of linearly independent vectors in the set.

Let's consider the vectors:

$  \mathbf{v}_1 =\begin{bmatrix} 1 \\ 2 \\ 1 \end{bmatrix}, \quad \mathbf{v}_2 =\begin{bmatrix} -1 \\ 0 \\ 1 \end{bmatrix}, \quad \mathbf{v}_3 =\begin{bmatrix} 2 \\ 2 \\ 3 \end{bmatrix}$ 

We want to determine if these vectors form a basis for $ \mathbb{R}^3$  and find their rank.

\textbf{Solution:}

To check if these vectors form a basis, we need to verify two
conditions:

\begin{enumerate}
\def\labelenumi{\arabic{enumi}.}
\item
  \textbf{Spanning the space:} We need to check if any vector in $ 
  \mathbb{R}^3$  can be expressed as a linear combination of $ 
  \mathbf{v}_1, \mathbf{v}_2, \mathbf{v}_3$  .
\item
  \textbf{Linear Independence:} We need to check if the vectors are
  linearly independent.
\end{enumerate}

First, let's set up the augmented matrix and perform Gaussian
elimination:

$  A \textbar{} B =\left[ \begin{array}{ccc|c} 1 & -1 & 2 & 0 \\ 2 & 0 & 2 & 0 \\ 1 & 1 & 3 & 0 \end{array} \right] $ 

Performing row operations:

\begin{enumerate}
\def\labelenumi{\arabic{enumi}.}
\item
  \textbf{Row 2 - 2 * Row 1:} $ 
  \left[ \begin{array}{ccc|c} 1 & -1 & 2 & 0 \\ 0 & 2 & -2 & 0 \\ 1 & 1 & 3 & 0 \end{array} \right] $ 
\item
  \textbf{Row 3 - Row 1:} $ 
  \left[ \begin{array}{ccc|c} 1 & -1 & 2 & 0 \\ 0 & 2 & -2 & 0 \\ 0 & 2 & 1 & 0 \end{array} \right] $ 
\item
  \textbf{Row 3 - Row 2:} $ 
  \left[ \begin{array}{ccc|c} 1 & -1 & 2 & 0 \\ 0 & 2 & -2 & 0 \\ 0 & 0 & 3 & 0 \end{array} \right] $ 
\end{enumerate}

Now, the system is in row-echelon form. We can see that there are no
free variables, and the system has a unique solution, indicating that
the vectors are linearly independent.

Next, let's check if they span $  \mathbb{R}^3$  . The matrix is of
full rank ($  3$  ), so the vectors span $  \mathbb{R}^3$  .

Therefore, the vectors $  \mathbf{v}\_1, \mathbf{v}\_2, \mathbf{v}\_3$ 
form a basis for $  \mathbb{R}^3$  , and their rank is $  3$  .

    \begin{tcolorbox}[breakable, size=fbox, boxrule=1pt, pad at break*=1mm,colback=cellbackground, colframe=cellborder]
\prompt{In}{incolor}{ }{\boxspacing}
\begin{Verbatim}[commandchars=\\\{\}]

\end{Verbatim}
\end{tcolorbox}

    \hypertarget{what-is-reduced-row-echelon-form}{%
\subsubsection{What is Reduced Row Echelon
Form}\label{what-is-reduced-row-echelon-form}}

    \textbf{Reduced Row Echelon Form (RREF):}

Reduced Row Echelon Form (RREF) is a further refinement of the row
echelon form. A matrix is in reduced row echelon form if it satisfies
the following additional conditions:

\begin{enumerate}
\def\labelenumi{\arabic{enumi}.}
\tightlist
\item
  It is in row echelon form.
\item
  The leading entry in each nonzero row is 1.
\item
  The leading 1 in each nonzero row is the only nonzero entry in its
  column.
\end{enumerate}

The reduced row echelon form is unique for a given matrix. It is often
denoted as ( \text{RREF}(A) ) or ( \text{rref}(A) ).

Here's an example in Markdown with LaTeX math mode:

\textbf{Example:}

Consider the matrix:

$  A =\begin{bmatrix} 2 & 1 & -1 & 4 \\ 4 & 2 & 1 & 7 \\ -2 & 1 & 2 & -1 \end{bmatrix}$ 

Let's find the reduced row echelon form using Gaussian elimination:

\begin{enumerate}
\def\labelenumi{\arabic{enumi}.}
\item
  \textbf{Row 2 - 2 * Row 1:} $ \begin{bmatrix} 2 & 1 & -1 & 4 \\ 0 & 0 & 3 & -1 \\ -2 & 1 & 2 & -1 \end{bmatrix} $ 
\item
  \textbf{Row 3 + Row 1:} $ \begin{bmatrix} 2 & 1 & -1 & 4 \\ 0 & 0 & 3 & -1 \\ 0 & 2 & 1 & 3 \end{bmatrix} $ 
\item
  \textbf{Row 3 - 2 * Row 2:} $ \begin{bmatrix} 2 & 1 & -1 & 4 \\ 0 & 0 & 3 & -1 \\ 0 & 2 & 1 & 3 \end{bmatrix} $ 
\item
  \textbf{Row 3/2:} $ \begin{bmatrix} 2 & 1 & -1 & 4 \\ 0 & 0 & 3 & -1 \\ 0 & 1 & 0.5 & 1.5 \end{bmatrix}  $ 
\item
  \textbf{Row 1 - Row 2:} $ \begin{bmatrix} 2 & 1 & -1 & 4 \\ 0 & 0 & 3 & -1 \\ 0 & 1 & 0.5 & 1.5 \end{bmatrix}  $ 
\item
  \textbf{Row 1/2:} $ \begin{bmatrix} 1 & 0.5 & -0.5 & 2 \\ 0 & 0 & 3 & -1 \\ 0 & 1 & 0.5 & 1.5 \end{bmatrix} $ 
\item
  \textbf{Row 2/3:} $ \begin{bmatrix} 1 & 0.5 & -0.5 & 2 \\ 0 & 0 & 1 & -1/3 \\ 0 & 1 & 0.5 & 1.5 \end{bmatrix} $ 
\item
  \textbf{Row 1 - 0.5 * Row 2:} $ \begin{bmatrix} 1 & 0.5 & 0 & 2.5 \\ 0 & 0 & 1 & -1/3 \\ 0 & 1 & 0.5 & 1.5 \end{bmatrix} $ 
\item
  \textbf{Row 2 + 0.5 * Row 3:} $ \begin{bmatrix} 1 & 0.5 & 0 & 2.5 \\ 0 & 1 & 0.25 & 0.75 \\ 0 & 1 & 0.5 & 1.5 \end{bmatrix} $ 
\item
  \textbf{Row 3 - Row 2:} $ \begin{bmatrix} 1 & 0.5 & 0 & 2.5 \\ 0 & 1 & 0.25 & 0.75 \\ 0 & 0 & 0.25 & 0.75 \end{bmatrix}  $ 
\item
  \textbf{Row 3/0.25:} $ \begin{bmatrix} 1 & 0.5 & 0 & 2.5 \\ 0 & 1 & 0.25 & 0.75 \\ 0 & 0 & 1 & 3 \end{bmatrix} $ 
\item
  \textbf{Row 1 - 0.5 * Row 2:} $ \begin{bmatrix} 1 & 0 & -0.125 & 2 \end{bmatrix} $ 
\end{enumerate}

Now, the matrix is in reduced row echelon form. The system of equations
corresponding to this matrix is:

$  x - 0.125z = 2 $ 

$  y + 0.25z = 0.75 $ 

$  z = 3 $ 

This system is consistent and has a unique solution. Therefore, the
reduced row echelon form provides a simplified representation of the
original system of equations.

    \begin{tcolorbox}[breakable, size=fbox, boxrule=1pt, pad at break*=1mm,colback=cellbackground, colframe=cellborder]
\prompt{In}{incolor}{ }{\boxspacing}
\begin{Verbatim}[commandchars=\\\{\}]

\end{Verbatim}
\end{tcolorbox}

    \hypertarget{image-and-kernel-of-a-transfoemration}{%
\subsubsection{Image and Kernel of a
Transfoemration}\label{image-and-kernel-of-a-transfoemration}}

    \textbf{Image and Kernel of a Transformation:}

In linear algebra, the image (or range) and kernel (or null space) are
two important subspaces associated with a linear transformation.

\begin{enumerate}
\def\labelenumi{\arabic{enumi}.}
\tightlist
\item
  \textbf{Image (Range):}

  \begin{itemize}
  \tightlist
  \item
    The image of a linear transformation $  T: V \rightarrow W$  is the
    set of all possible outputs or vectors in the codomain $  W$  that
    can be obtained by applying the transformation to vectors in the
    domain $  V$  .
  \item
    Mathematically, it is denoted as $  \text{Im}(T) $  or $ 
    \text{Range}(T) $  , and it is defined as: $  \text{Im}(T) = \{
    T(\mathbf{v}) \mid \mathbf{v} \in V \} $ 
  \end{itemize}
\item
  \textbf{Kernel (Null Space):}

  \begin{itemize}
  \tightlist
  \item
    The kernel of a linear transformation $  T: V \rightarrow W$  is the
    set of all vectors in the domain $  V$  that map to the zero vector
    in the codomain $  W$  .
  \item
    Mathematically, it is denoted as $  \text{ker}(T) $  or $ 
    \text{Null}(T) $  , and it is defined as: $  \text{ker}(T) = \{
    \mathbf{v} \in V \mid T(\mathbf{v}) = \mathbf{0} \} $ 
  \end{itemize}
\end{enumerate}

\textbf{Example:}

Let's consider a linear transformation $  T: \mathbb{R}^3
\rightarrow \mathbb{R}^2$  defined by the matrix:

$  A =\begin{bmatrix} 1 & 2 & -1 \\ 0 & 1 & 3 \end{bmatrix}$ 

The transformation is given by $  T(\mathbf{v}) = A\mathbf{v}$  .

\textbf{Image:} To find the image, we need to determine all possible
outputs by applying $  T$  to vectors in $  \mathbb{R}^3$  . The
image is the span of the column vectors of $  A$  :

$  \text{Im}(T) = \text{span}\left(
\begin{bmatrix} 1 \\ 0 \end{bmatrix}
,
\begin{bmatrix} 2 \\ 1 \end{bmatrix}
\right) $ 

\textbf{Kernel:} To find the kernel, we need to find vectors in $ 
\mathbb{R}^3$  that map to the zero vector in $  \mathbb{R}^2$  .
This involves solving the homogeneous system of equations $  A\mathbf{v}
= \mathbf{0}$  :

$  A\mathbf{v} =\begin{bmatrix} 1 & 2 & -1 \\ 0 & 1 & 3 \end{bmatrix} \begin{bmatrix} x \\ y \\ z \end{bmatrix}=\begin{bmatrix} 0 \\ 0 \end{bmatrix}$ 

Solving this system will give us the vectors in the kernel.

Let's solve it:

$ 
\begin{bmatrix} 1 & 2 & -1 \\ 0 & 1 & 3 \end{bmatrix} \begin{bmatrix} x \\ y \\ z \end{bmatrix}=\begin{bmatrix} 0 \\ 0 \end{bmatrix}$ 

\begin{enumerate}
\def\labelenumi{\arabic{enumi}.}
\tightlist
\item
  $  x + 2y - z = 0$ 
\item
  $  y + 3z = 0$ 
\end{enumerate}

The solutions are $  x = -2t$  , $  y = -3t$  , $  z = t$  , where $ 
t$  is any real number. Therefore, the kernel is:

$  \text{ker}(T) = \text{span}\left(\begin{bmatrix} -2 \\ -3 \\ 1 \end{bmatrix}\right) $ 

    \begin{tcolorbox}[breakable, size=fbox, boxrule=1pt, pad at break*=1mm,colback=cellbackground, colframe=cellborder]
\prompt{In}{incolor}{ }{\boxspacing}
\begin{Verbatim}[commandchars=\\\{\}]

\end{Verbatim}
\end{tcolorbox}


    % Add a bibliography block to the postdoc
    
    
    
\end{document}
