\documentclass[11pt]{article}

    \usepackage[breakable]{tcolorbox}
    \usepackage{parskip} % Stop auto-indenting (to mimic markdown behaviour)
    

    % Basic figure setup, for now with no caption control since it's done
    % automatically by Pandoc (which extracts ![](path) syntax from Markdown).
    \usepackage{graphicx}
    % Maintain compatibility with old templates. Remove in nbconvert 6.0
    \let\Oldincludegraphics\includegraphics
    % Ensure that by default, figures have no caption (until we provide a
    % proper Figure object with a Caption API and a way to capture that
    % in the conversion process - todo).
    \usepackage{caption}
    \DeclareCaptionFormat{nocaption}{}
    \captionsetup{format=nocaption,aboveskip=0pt,belowskip=0pt}

    \usepackage{float}
    \floatplacement{figure}{H} % forces figures to be placed at the correct location
    \usepackage{xcolor} % Allow colors to be defined
    \usepackage{enumerate} % Needed for markdown enumerations to work
    \usepackage{geometry} % Used to adjust the document margins
    \usepackage{amsmath} % Equations
    \usepackage{amssymb} % Equations
    \usepackage{textcomp} % defines textquotesingle
    % Hack from http://tex.stackexchange.com/a/47451/13684:
    \AtBeginDocument{%
        \def\PYZsq{\textquotesingle}% Upright quotes in Pygmentized code
    }
    \usepackage{upquote} % Upright quotes for verbatim code
    \usepackage{eurosym} % defines \euro

    \usepackage{iftex}
    \ifPDFTeX
        \usepackage[T1]{fontenc}
        \IfFileExists{alphabeta.sty}{
              \usepackage{alphabeta}
          }{
              \usepackage[mathletters]{ucs}
              \usepackage[utf8x]{inputenc}
          }
    \else
        \usepackage{fontspec}
        \usepackage{unicode-math}
    \fi

    \usepackage{fancyvrb} % verbatim replacement that allows latex
    \usepackage[Export]{adjustbox} % Used to constrain images to a maximum size
    \adjustboxset{max size={0.9\linewidth}{0.9\paperheight}}

    % The hyperref package gives us a pdf with properly built
    % internal navigation ('pdf bookmarks' for the table of contents,
    % internal cross-reference links, web links for URLs, etc.)
    \usepackage{hyperref}
    % The default LaTeX title has an obnoxious amount of whitespace. By default,
    % titling removes some of it. It also provides customization options.
    \usepackage{titling}
    \usepackage{longtable} % longtable support required by pandoc >1.10
    \usepackage{booktabs}  % table support for pandoc > 1.12.2
    \usepackage{array}     % table support for pandoc >= 2.11.3
    \usepackage{calc}      % table minipage width calculation for pandoc >= 2.11.1
    \usepackage[inline]{enumitem} % IRkernel/repr support (it uses the enumerate* environment)
    \usepackage[normalem]{ulem} % ulem is needed to support strikethroughs (\sout)
                                % normalem makes italics be italics, not underlines
    \usepackage{mathrsfs}
    

    
    % Colors for the hyperref package
    \definecolor{urlcolor}{rgb}{0,.145,.698}
    \definecolor{linkcolor}{rgb}{.71,0.21,0.01}
    \definecolor{citecolor}{rgb}{.12,.54,.11}

    % ANSI colors
    \definecolor{ansi-black}{HTML}{3E424D}
    \definecolor{ansi-black-intense}{HTML}{282C36}
    \definecolor{ansi-red}{HTML}{E75C58}
    \definecolor{ansi-red-intense}{HTML}{B22B31}
    \definecolor{ansi-green}{HTML}{00A250}
    \definecolor{ansi-green-intense}{HTML}{007427}
    \definecolor{ansi-yellow}{HTML}{DDB62B}
    \definecolor{ansi-yellow-intense}{HTML}{B27D12}
    \definecolor{ansi-blue}{HTML}{208FFB}
    \definecolor{ansi-blue-intense}{HTML}{0065CA}
    \definecolor{ansi-magenta}{HTML}{D160C4}
    \definecolor{ansi-magenta-intense}{HTML}{A03196}
    \definecolor{ansi-cyan}{HTML}{60C6C8}
    \definecolor{ansi-cyan-intense}{HTML}{258F8F}
    \definecolor{ansi-white}{HTML}{C5C1B4}
    \definecolor{ansi-white-intense}{HTML}{A1A6B2}
    \definecolor{ansi-default-inverse-fg}{HTML}{FFFFFF}
    \definecolor{ansi-default-inverse-bg}{HTML}{000000}

    % common color for the border for error outputs.
    \definecolor{outerrorbackground}{HTML}{FFDFDF}

    % commands and environments needed by pandoc snippets
    % extracted from the output of `pandoc -s`
    \providecommand{\tightlist}{%
      \setlength{\itemsep}{0pt}\setlength{\parskip}{0pt}}
    \DefineVerbatimEnvironment{Highlighting}{Verbatim}{commandchars=\\\{\}}
    % Add ',fontsize=\small' for more characters per line
    \newenvironment{Shaded}{}{}
    \newcommand{\KeywordTok}[1]{\textcolor[rgb]{0.00,0.44,0.13}{\textbf{{#1}}}}
    \newcommand{\DataTypeTok}[1]{\textcolor[rgb]{0.56,0.13,0.00}{{#1}}}
    \newcommand{\DecValTok}[1]{\textcolor[rgb]{0.25,0.63,0.44}{{#1}}}
    \newcommand{\BaseNTok}[1]{\textcolor[rgb]{0.25,0.63,0.44}{{#1}}}
    \newcommand{\FloatTok}[1]{\textcolor[rgb]{0.25,0.63,0.44}{{#1}}}
    \newcommand{\CharTok}[1]{\textcolor[rgb]{0.25,0.44,0.63}{{#1}}}
    \newcommand{\StringTok}[1]{\textcolor[rgb]{0.25,0.44,0.63}{{#1}}}
    \newcommand{\CommentTok}[1]{\textcolor[rgb]{0.38,0.63,0.69}{\textit{{#1}}}}
    \newcommand{\OtherTok}[1]{\textcolor[rgb]{0.00,0.44,0.13}{{#1}}}
    \newcommand{\AlertTok}[1]{\textcolor[rgb]{1.00,0.00,0.00}{\textbf{{#1}}}}
    \newcommand{\FunctionTok}[1]{\textcolor[rgb]{0.02,0.16,0.49}{{#1}}}
    \newcommand{\RegionMarkerTok}[1]{{#1}}
    \newcommand{\ErrorTok}[1]{\textcolor[rgb]{1.00,0.00,0.00}{\textbf{{#1}}}}
    \newcommand{\NormalTok}[1]{{#1}}
    
    % Additional commands for more recent versions of Pandoc
    \newcommand{\ConstantTok}[1]{\textcolor[rgb]{0.53,0.00,0.00}{{#1}}}
    \newcommand{\SpecialCharTok}[1]{\textcolor[rgb]{0.25,0.44,0.63}{{#1}}}
    \newcommand{\VerbatimStringTok}[1]{\textcolor[rgb]{0.25,0.44,0.63}{{#1}}}
    \newcommand{\SpecialStringTok}[1]{\textcolor[rgb]{0.73,0.40,0.53}{{#1}}}
    \newcommand{\ImportTok}[1]{{#1}}
    \newcommand{\DocumentationTok}[1]{\textcolor[rgb]{0.73,0.13,0.13}{\textit{{#1}}}}
    \newcommand{\AnnotationTok}[1]{\textcolor[rgb]{0.38,0.63,0.69}{\textbf{\textit{{#1}}}}}
    \newcommand{\CommentVarTok}[1]{\textcolor[rgb]{0.38,0.63,0.69}{\textbf{\textit{{#1}}}}}
    \newcommand{\VariableTok}[1]{\textcolor[rgb]{0.10,0.09,0.49}{{#1}}}
    \newcommand{\ControlFlowTok}[1]{\textcolor[rgb]{0.00,0.44,0.13}{\textbf{{#1}}}}
    \newcommand{\OperatorTok}[1]{\textcolor[rgb]{0.40,0.40,0.40}{{#1}}}
    \newcommand{\BuiltInTok}[1]{{#1}}
    \newcommand{\ExtensionTok}[1]{{#1}}
    \newcommand{\PreprocessorTok}[1]{\textcolor[rgb]{0.74,0.48,0.00}{{#1}}}
    \newcommand{\AttributeTok}[1]{\textcolor[rgb]{0.49,0.56,0.16}{{#1}}}
    \newcommand{\InformationTok}[1]{\textcolor[rgb]{0.38,0.63,0.69}{\textbf{\textit{{#1}}}}}
    \newcommand{\WarningTok}[1]{\textcolor[rgb]{0.38,0.63,0.69}{\textbf{\textit{{#1}}}}}
    
    
    % Define a nice break command that doesn't care if a line doesn't already
    % exist.
    \def\br{\hspace*{\fill} \\* }
    % Math Jax compatibility definitions
    \def\gt{>}
    \def\lt{<}
    \let\Oldtex\TeX
    \let\Oldlatex\LaTeX
    \renewcommand{\TeX}{\textrm{\Oldtex}}
    \renewcommand{\LaTeX}{\textrm{\Oldlatex}}
    % Document parameters
    % Document title
    \title{Linear Algebra Training Quiz}
    
    
    
    
    
% Pygments definitions
\makeatletter
\def\PY@reset{\let\PY@it=\relax \let\PY@bf=\relax%
    \let\PY@ul=\relax \let\PY@tc=\relax%
    \let\PY@bc=\relax \let\PY@ff=\relax}
\def\PY@tok#1{\csname PY@tok@#1\endcsname}
\def\PY@toks#1+{\ifx\relax#1\empty\else%
    \PY@tok{#1}\expandafter\PY@toks\fi}
\def\PY@do#1{\PY@bc{\PY@tc{\PY@ul{%
    \PY@it{\PY@bf{\PY@ff{#1}}}}}}}
\def\PY#1#2{\PY@reset\PY@toks#1+\relax+\PY@do{#2}}

\@namedef{PY@tok@w}{\def\PY@tc##1{\textcolor[rgb]{0.73,0.73,0.73}{##1}}}
\@namedef{PY@tok@c}{\let\PY@it=\textit\def\PY@tc##1{\textcolor[rgb]{0.24,0.48,0.48}{##1}}}
\@namedef{PY@tok@cp}{\def\PY@tc##1{\textcolor[rgb]{0.61,0.40,0.00}{##1}}}
\@namedef{PY@tok@k}{\let\PY@bf=\textbf\def\PY@tc##1{\textcolor[rgb]{0.00,0.50,0.00}{##1}}}
\@namedef{PY@tok@kp}{\def\PY@tc##1{\textcolor[rgb]{0.00,0.50,0.00}{##1}}}
\@namedef{PY@tok@kt}{\def\PY@tc##1{\textcolor[rgb]{0.69,0.00,0.25}{##1}}}
\@namedef{PY@tok@o}{\def\PY@tc##1{\textcolor[rgb]{0.40,0.40,0.40}{##1}}}
\@namedef{PY@tok@ow}{\let\PY@bf=\textbf\def\PY@tc##1{\textcolor[rgb]{0.67,0.13,1.00}{##1}}}
\@namedef{PY@tok@nb}{\def\PY@tc##1{\textcolor[rgb]{0.00,0.50,0.00}{##1}}}
\@namedef{PY@tok@nf}{\def\PY@tc##1{\textcolor[rgb]{0.00,0.00,1.00}{##1}}}
\@namedef{PY@tok@nc}{\let\PY@bf=\textbf\def\PY@tc##1{\textcolor[rgb]{0.00,0.00,1.00}{##1}}}
\@namedef{PY@tok@nn}{\let\PY@bf=\textbf\def\PY@tc##1{\textcolor[rgb]{0.00,0.00,1.00}{##1}}}
\@namedef{PY@tok@ne}{\let\PY@bf=\textbf\def\PY@tc##1{\textcolor[rgb]{0.80,0.25,0.22}{##1}}}
\@namedef{PY@tok@nv}{\def\PY@tc##1{\textcolor[rgb]{0.10,0.09,0.49}{##1}}}
\@namedef{PY@tok@no}{\def\PY@tc##1{\textcolor[rgb]{0.53,0.00,0.00}{##1}}}
\@namedef{PY@tok@nl}{\def\PY@tc##1{\textcolor[rgb]{0.46,0.46,0.00}{##1}}}
\@namedef{PY@tok@ni}{\let\PY@bf=\textbf\def\PY@tc##1{\textcolor[rgb]{0.44,0.44,0.44}{##1}}}
\@namedef{PY@tok@na}{\def\PY@tc##1{\textcolor[rgb]{0.41,0.47,0.13}{##1}}}
\@namedef{PY@tok@nt}{\let\PY@bf=\textbf\def\PY@tc##1{\textcolor[rgb]{0.00,0.50,0.00}{##1}}}
\@namedef{PY@tok@nd}{\def\PY@tc##1{\textcolor[rgb]{0.67,0.13,1.00}{##1}}}
\@namedef{PY@tok@s}{\def\PY@tc##1{\textcolor[rgb]{0.73,0.13,0.13}{##1}}}
\@namedef{PY@tok@sd}{\let\PY@it=\textit\def\PY@tc##1{\textcolor[rgb]{0.73,0.13,0.13}{##1}}}
\@namedef{PY@tok@si}{\let\PY@bf=\textbf\def\PY@tc##1{\textcolor[rgb]{0.64,0.35,0.47}{##1}}}
\@namedef{PY@tok@se}{\let\PY@bf=\textbf\def\PY@tc##1{\textcolor[rgb]{0.67,0.36,0.12}{##1}}}
\@namedef{PY@tok@sr}{\def\PY@tc##1{\textcolor[rgb]{0.64,0.35,0.47}{##1}}}
\@namedef{PY@tok@ss}{\def\PY@tc##1{\textcolor[rgb]{0.10,0.09,0.49}{##1}}}
\@namedef{PY@tok@sx}{\def\PY@tc##1{\textcolor[rgb]{0.00,0.50,0.00}{##1}}}
\@namedef{PY@tok@m}{\def\PY@tc##1{\textcolor[rgb]{0.40,0.40,0.40}{##1}}}
\@namedef{PY@tok@gh}{\let\PY@bf=\textbf\def\PY@tc##1{\textcolor[rgb]{0.00,0.00,0.50}{##1}}}
\@namedef{PY@tok@gu}{\let\PY@bf=\textbf\def\PY@tc##1{\textcolor[rgb]{0.50,0.00,0.50}{##1}}}
\@namedef{PY@tok@gd}{\def\PY@tc##1{\textcolor[rgb]{0.63,0.00,0.00}{##1}}}
\@namedef{PY@tok@gi}{\def\PY@tc##1{\textcolor[rgb]{0.00,0.52,0.00}{##1}}}
\@namedef{PY@tok@gr}{\def\PY@tc##1{\textcolor[rgb]{0.89,0.00,0.00}{##1}}}
\@namedef{PY@tok@ge}{\let\PY@it=\textit}
\@namedef{PY@tok@gs}{\let\PY@bf=\textbf}
\@namedef{PY@tok@gp}{\let\PY@bf=\textbf\def\PY@tc##1{\textcolor[rgb]{0.00,0.00,0.50}{##1}}}
\@namedef{PY@tok@go}{\def\PY@tc##1{\textcolor[rgb]{0.44,0.44,0.44}{##1}}}
\@namedef{PY@tok@gt}{\def\PY@tc##1{\textcolor[rgb]{0.00,0.27,0.87}{##1}}}
\@namedef{PY@tok@err}{\def\PY@bc##1{{\setlength{\fboxsep}{\string -\fboxrule}\fcolorbox[rgb]{1.00,0.00,0.00}{1,1,1}{\strut ##1}}}}
\@namedef{PY@tok@kc}{\let\PY@bf=\textbf\def\PY@tc##1{\textcolor[rgb]{0.00,0.50,0.00}{##1}}}
\@namedef{PY@tok@kd}{\let\PY@bf=\textbf\def\PY@tc##1{\textcolor[rgb]{0.00,0.50,0.00}{##1}}}
\@namedef{PY@tok@kn}{\let\PY@bf=\textbf\def\PY@tc##1{\textcolor[rgb]{0.00,0.50,0.00}{##1}}}
\@namedef{PY@tok@kr}{\let\PY@bf=\textbf\def\PY@tc##1{\textcolor[rgb]{0.00,0.50,0.00}{##1}}}
\@namedef{PY@tok@bp}{\def\PY@tc##1{\textcolor[rgb]{0.00,0.50,0.00}{##1}}}
\@namedef{PY@tok@fm}{\def\PY@tc##1{\textcolor[rgb]{0.00,0.00,1.00}{##1}}}
\@namedef{PY@tok@vc}{\def\PY@tc##1{\textcolor[rgb]{0.10,0.09,0.49}{##1}}}
\@namedef{PY@tok@vg}{\def\PY@tc##1{\textcolor[rgb]{0.10,0.09,0.49}{##1}}}
\@namedef{PY@tok@vi}{\def\PY@tc##1{\textcolor[rgb]{0.10,0.09,0.49}{##1}}}
\@namedef{PY@tok@vm}{\def\PY@tc##1{\textcolor[rgb]{0.10,0.09,0.49}{##1}}}
\@namedef{PY@tok@sa}{\def\PY@tc##1{\textcolor[rgb]{0.73,0.13,0.13}{##1}}}
\@namedef{PY@tok@sb}{\def\PY@tc##1{\textcolor[rgb]{0.73,0.13,0.13}{##1}}}
\@namedef{PY@tok@sc}{\def\PY@tc##1{\textcolor[rgb]{0.73,0.13,0.13}{##1}}}
\@namedef{PY@tok@dl}{\def\PY@tc##1{\textcolor[rgb]{0.73,0.13,0.13}{##1}}}
\@namedef{PY@tok@s2}{\def\PY@tc##1{\textcolor[rgb]{0.73,0.13,0.13}{##1}}}
\@namedef{PY@tok@sh}{\def\PY@tc##1{\textcolor[rgb]{0.73,0.13,0.13}{##1}}}
\@namedef{PY@tok@s1}{\def\PY@tc##1{\textcolor[rgb]{0.73,0.13,0.13}{##1}}}
\@namedef{PY@tok@mb}{\def\PY@tc##1{\textcolor[rgb]{0.40,0.40,0.40}{##1}}}
\@namedef{PY@tok@mf}{\def\PY@tc##1{\textcolor[rgb]{0.40,0.40,0.40}{##1}}}
\@namedef{PY@tok@mh}{\def\PY@tc##1{\textcolor[rgb]{0.40,0.40,0.40}{##1}}}
\@namedef{PY@tok@mi}{\def\PY@tc##1{\textcolor[rgb]{0.40,0.40,0.40}{##1}}}
\@namedef{PY@tok@il}{\def\PY@tc##1{\textcolor[rgb]{0.40,0.40,0.40}{##1}}}
\@namedef{PY@tok@mo}{\def\PY@tc##1{\textcolor[rgb]{0.40,0.40,0.40}{##1}}}
\@namedef{PY@tok@ch}{\let\PY@it=\textit\def\PY@tc##1{\textcolor[rgb]{0.24,0.48,0.48}{##1}}}
\@namedef{PY@tok@cm}{\let\PY@it=\textit\def\PY@tc##1{\textcolor[rgb]{0.24,0.48,0.48}{##1}}}
\@namedef{PY@tok@cpf}{\let\PY@it=\textit\def\PY@tc##1{\textcolor[rgb]{0.24,0.48,0.48}{##1}}}
\@namedef{PY@tok@c1}{\let\PY@it=\textit\def\PY@tc##1{\textcolor[rgb]{0.24,0.48,0.48}{##1}}}
\@namedef{PY@tok@cs}{\let\PY@it=\textit\def\PY@tc##1{\textcolor[rgb]{0.24,0.48,0.48}{##1}}}

\def\PYZbs{\char`\\}
\def\PYZus{\char`\_}
\def\PYZob{\char`\{}
\def\PYZcb{\char`\}}
\def\PYZca{\char`\^}
\def\PYZam{\char`\&}
\def\PYZlt{\char`\<}
\def\PYZgt{\char`\>}
\def\PYZsh{\char`\#}
\def\PYZpc{\char`\%}
\def\PYZdl{\char`$}
\def\PYZhy{\char`\-}
\def\PYZsq{\char`\'}
\def\PYZdq{\char`\"}
\def\PYZti{\char`\~}
% for compatibility with earlier versions
\def\PYZat{@}
\def\PYZlb{[}
\def\PYZrb{]}
\makeatother


    % For linebreaks inside Verbatim environment from package fancyvrb. 
    \makeatletter
        \newbox\Wrappedcontinuationbox 
        \newbox\Wrappedvisiblespacebox 
        \newcommand*\Wrappedvisiblespace {\textcolor{red}{\textvisiblespace}} 
        \newcommand*\Wrappedcontinuationsymbol {\textcolor{red}{\llap{\tiny$\m@th\hookrightarrow$}}} 
        \newcommand*\Wrappedcontinuationindent {3ex } 
        \newcommand*\Wrappedafterbreak {\kern\Wrappedcontinuationindent\copy\Wrappedcontinuationbox} 
        % Take advantage of the already applied Pygments mark-up to insert 
        % potential linebreaks for TeX processing. 
        %        {, <, #, %, $, ' and ": go to next line. 
        %        _, }, ^, &, >, - and ~: stay at end of broken line. 
        % Use of \textquotesingle for straight quote. 
        \newcommand*\Wrappedbreaksatspecials {% 
            \def\PYGZus{\discretionary{\char`\_}{\Wrappedafterbreak}{\char`\_}}% 
            \def\PYGZob{\discretionary{}{\Wrappedafterbreak\char`\{}{\char`\{}}% 
            \def\PYGZcb{\discretionary{\char`\}}{\Wrappedafterbreak}{\char`\}}}% 
            \def\PYGZca{\discretionary{\char`\^}{\Wrappedafterbreak}{\char`\^}}% 
            \def\PYGZam{\discretionary{\char`\&}{\Wrappedafterbreak}{\char`\&}}% 
            \def\PYGZlt{\discretionary{}{\Wrappedafterbreak\char`\<}{\char`\<}}% 
            \def\PYGZgt{\discretionary{\char`\>}{\Wrappedafterbreak}{\char`\>}}% 
            \def\PYGZsh{\discretionary{}{\Wrappedafterbreak\char`\#}{\char`\#}}% 
            \def\PYGZpc{\discretionary{}{\Wrappedafterbreak\char`\%}{\char`\%}}% 
            \def\PYGZdl{\discretionary{}{\Wrappedafterbreak\char`$}{\char`$}}% 
            \def\PYGZhy{\discretionary{\char`\-}{\Wrappedafterbreak}{\char`\-}}% 
            \def\PYGZsq{\discretionary{}{\Wrappedafterbreak\textquotesingle}{\textquotesingle}}% 
            \def\PYGZdq{\discretionary{}{\Wrappedafterbreak\char`\"}{\char`\"}}% 
            \def\PYGZti{\discretionary{\char`\~}{\Wrappedafterbreak}{\char`\~}}% 
        } 
        % Some characters . , ; ? ! / are not pygmentized. 
        % This macro makes them "active" and they will insert potential linebreaks 
        \newcommand*\Wrappedbreaksatpunct {% 
            \lccode`\~`\.\lowercase{\def~}{\discretionary{\hbox{\char`\.}}{\Wrappedafterbreak}{\hbox{\char`\.}}}% 
            \lccode`\~`\,\lowercase{\def~}{\discretionary{\hbox{\char`\,}}{\Wrappedafterbreak}{\hbox{\char`\,}}}% 
            \lccode`\~`\;\lowercase{\def~}{\discretionary{\hbox{\char`\;}}{\Wrappedafterbreak}{\hbox{\char`\;}}}% 
            \lccode`\~`\:\lowercase{\def~}{\discretionary{\hbox{\char`\:}}{\Wrappedafterbreak}{\hbox{\char`\:}}}% 
            \lccode`\~`\?\lowercase{\def~}{\discretionary{\hbox{\char`\?}}{\Wrappedafterbreak}{\hbox{\char`\?}}}% 
            \lccode`\~`\!\lowercase{\def~}{\discretionary{\hbox{\char`\!}}{\Wrappedafterbreak}{\hbox{\char`\!}}}% 
            \lccode`\~`\/\lowercase{\def~}{\discretionary{\hbox{\char`\/}}{\Wrappedafterbreak}{\hbox{\char`\/}}}% 
            \catcode`\.\active
            \catcode`\,\active 
            \catcode`\;\active
            \catcode`\:\active
            \catcode`\?\active
            \catcode`\!\active
            \catcode`\/\active 
            \lccode`\~`\~ 	
        }
    \makeatother

    \let\OriginalVerbatim=\Verbatim
    \makeatletter
    \renewcommand{\Verbatim}[1][1]{%
        %\parskip\z@skip
        \sbox\Wrappedcontinuationbox {\Wrappedcontinuationsymbol}%
        \sbox\Wrappedvisiblespacebox {\FV@SetupFont\Wrappedvisiblespace}%
        \def\FancyVerbFormatLine ##1{\hsize\linewidth
            \vtop{\raggedright\hyphenpenalty\z@\exhyphenpenalty\z@
                \doublehyphendemerits\z@\finalhyphendemerits\z@
                \strut ##1\strut}%
        }%
        % If the linebreak is at a space, the latter will be displayed as visible
        % space at end of first line, and a continuation symbol starts next line.
        % Stretch/shrink are however usually zero for typewriter font.
        \def\FV@Space {%
            \nobreak\hskip\z@ plus\fontdimen3\font minus\fontdimen4\font
            \discretionary{\copy\Wrappedvisiblespacebox}{\Wrappedafterbreak}
            {\kern\fontdimen2\font}%
        }%
        
        % Allow breaks at special characters using \PYG... macros.
        \Wrappedbreaksatspecials
        % Breaks at punctuation characters . , ; ? ! and / need catcode=\active 	
        \OriginalVerbatim[#1,codes*=\Wrappedbreaksatpunct]%
    }
    \makeatother

    % Exact colors from NB
    \definecolor{incolor}{HTML}{303F9F}
    \definecolor{outcolor}{HTML}{D84315}
    \definecolor{cellborder}{HTML}{CFCFCF}
    \definecolor{cellbackground}{HTML}{F7F7F7}
    
    % prompt
    \makeatletter
    \newcommand{\boxspacing}{\kern\kvtcb@left@rule\kern\kvtcb@boxsep}
    \makeatother
    \newcommand{\prompt}[4]{
        {\ttfamily\llap{{\color{#2}[#3]:\hspace{3pt}#4}}\vspace{-\baselineskip}}
    }
    

    
    % Prevent overflowing lines due to hard-to-break entities
    \sloppy 
    % Setup hyperref package
    \hypersetup{
      breaklinks=true,  % so long urls are correctly broken across lines
      colorlinks=true,
      urlcolor=urlcolor,
      linkcolor=linkcolor,
      citecolor=citecolor,
      }
    % Slightly bigger margins than the latex defaults
    
    \geometry{verbose,tmargin=1in,bmargin=1in,lmargin=1in,rmargin=1in}
    
    

\begin{document}
    
    \maketitle
    
    

    
    $\textbf{Question 1}$\\
~\\
For which value(s) of the parameter b is the real matrix
$A= \begin{bmatrix} 1 & b \\b & 9 \end{bmatrix}$ positive definite ?\\
$\textbf{Answer :}$

    For the real matrix $A = \begin{bmatrix} 1 & b \\ b & 9 \end{bmatrix}$
to be positive definite, the following condition must be satisfied:

$ \det(A\_2) = \det\left(
\begin{bmatrix} 1 & b \\ b & 9 \end{bmatrix}
\right) = 9 - b^2 \textgreater{} 0 $

This leads to the inequality:

$ b^2 \textless{} 9 $

Therefore, the matrix $A$ is positive definite for all values of $b$
such that $-3 < b < 3$.

    $\textbf{Question 2}$\\
~\\
Which of the following matrices is symmetric positive definite?\\
~\\
$A=\begin{bmatrix} 1 & 2 \\ 3 & 1 \end{bmatrix}$
$B=\begin{bmatrix} -4 & -2 \\ -2 & -4 \end{bmatrix}$
$C=\begin{bmatrix} -2 & 1 & -1 \\ 1 & -2 & 1 \\ -1 & 1 & -2 \end{bmatrix}$
$D=\begin{bmatrix} 5 & 0 \\ 1 & 0 \end{bmatrix}$
$E=\begin{bmatrix} 4 & -2 \\ -2 & 4 \end{bmatrix}$\\
~\\
$\textbf{Answer :}$

    $\textbf{Matrix Symmetry and Positive Definiteness}$

$\textbf{Matrix $A$:}$ 1. Check symmetry:
$A^T = \begin{bmatrix} 1 & 3 \\ 2 & 1 \end{bmatrix} \neq A$
Conclusion: $A$ is not symmetric.

$\textbf{Matrix $B$:}$ 1. Check symmetry:
$B^T = \begin{bmatrix} -4 & -2 \\ -2 & -4 \end{bmatrix} = B$

\begin{enumerate}
\def\labelenumi{\arabic{enumi}.}
\setcounter{enumi}{1}
\tightlist
\item
  Find eigenvalues:
  $\text{det}(B - \lambda I) = \text{det}\left(\begin{bmatrix} -4-\lambda & -2 \\ -2 & -4-\lambda \end{bmatrix}\right) = 0$
  $(\lambda + 2)(\lambda + 6) = 0$ Eigenvalues:
  $\lambda_1 = -2, \lambda_2 = -6$ Conclusion: $B$ has both positive
  and negative eigenvalues.
\end{enumerate}

$\textbf{Matrix $C$:}$ 1. Check symmetry:
$C^T = \begin{bmatrix} -2 & 1 & -1 \\ 1 & -2 & 1 \\ -1 & 1 & -2 \end{bmatrix} = C$

\begin{enumerate}
\def\labelenumi{\arabic{enumi}.}
\setcounter{enumi}{1}
\tightlist
\item
  Find eigenvalues:
  $\text{det}(C - \lambda I) = \text{det}\left(\begin{bmatrix} -2-\lambda & 1 & -1 \\ 1 & -2-\lambda & 1 \\ -1 & 1 & -2-\lambda \end{bmatrix}\right) = 0$
  $(\lambda + 3)(\lambda + 3)(\lambda + 3) = 0$ Eigenvalues:
  $\lambda_1 = -3, \lambda_2 = -3, \lambda_3 = -3$ Conclusion: $C$
  has non-positive eigenvalues.
\end{enumerate}

$\textbf{Matrix $D$:}$ 1. Check symmetry:
$D^T = \begin{bmatrix} 5 & 1 \\ 0 & 0 \end{bmatrix} \neq D$
Conclusion: $D$ is not symmetric.

$\textbf{Matrix $E$:}$ 1. Check symmetry:
$E^T = \begin{bmatrix} 4 & -2 \\ -2 & 4 \end{bmatrix} = E$

\begin{enumerate}
\def\labelenumi{\arabic{enumi}.}
\setcounter{enumi}{1}
\tightlist
\item
  Find eigenvalues:
  $\text{det}(E - \lambda I) = \text{det}\left(\begin{bmatrix} 4-\lambda & -2 \\ -2 & 4-\lambda \end{bmatrix}\right) = 0$
  $(\lambda - 2)(\lambda - 6) = 0$ Eigenvalues:
  $\lambda_1 = 2, \lambda_2 = 6$ Conclusion: $E$ has both positive
  eigenvalues.
\end{enumerate}

In summary: - Matrix $A$ is not symmetric. - Matrices $B$, $C$,
and $D$ are not symmetric positive definite. - Matrix $E$ is
symmetric positive definite.

    $\textbf{Question 3}$\\
~\\
Is the matrix
$C=\begin{bmatrix} 2 & 2 & 2 \\ 2 & 2 & 2 \\ 2 & 2 & 0 \end{bmatrix}$\\
a. Not positive semidefinite. b. Positive semidefinite (but not positive
definite). c.~Positive definite. d.~Non symmetric. e. Not square.\\
~\\
$\textbf{Answer :}$

    The matrix $C $ is given by: $ C =
\begin{bmatrix} 2 & 2 & 2 \\ 2 & 2 & 2 \\ 2 & 2 & 0 \end{bmatrix}
$

The characteristic polynomial is obtained by solving the equation:

$ (2-\lambda)(\lambda^2 - 4\lambda + 16) = 0 $

The roots of this equation give the eigenvalues. Solving for
$\lambda $:

$ \lambda^2 - 4\lambda + 16 = 0 $

The discriminant ($b^2 - 4ac $) is $(-4)^2 - 4(1)(16) = 16 - 64
= -48 $, which is negative. Therefore, the quadratic equation has
complex conjugate roots.

The eigenvalues are complex numbers, and since there exists at least one
non-real eigenvalue, the matrix $C $ is not positive definite.

So the correct answer is:

\begin{enumerate}
\def\labelenumi{\alph{enumi}.}
\tightlist
\item
  Not positive semidefinite.
\end{enumerate}

    $\textbf{Question 4}$\\
~\\
Given the matrix $A=\begin{bmatrix} 2 & 2 \\ k & 4 \end{bmatrix}$ we
define the inner product $x^TAx$ where $x^T=(x_1,x_2)$.\\
For which value of k the associated quadratic form equals
$2x{_1}^2+4x_1x_2+4x{_2}^2$ ?\\
~\\
$\textbf{Answer :}$

    Given the matrix $A = \begin{bmatrix} 2 & 2 \\ k & 4 \end{bmatrix}$,
we define the inner product $x^TAx$ where $x^T = (x_1, x_2)$. For
which value of $k$ does the associated quadratic form equal
$2x_1^2 + 4x_1x_2 + 4x_2^2$?

The quadratic form $x^TAx$ is given by:
$x^TAx = \begin{bmatrix} x_1 & x_2 \end{bmatrix} \begin{bmatrix} 2 & 2 \\ k & 4 \end{bmatrix} \begin{bmatrix} x_1 \\ x_2 \end{bmatrix}$

Multiplying the matrices, we get:
$x^TAx = \begin{bmatrix} x_1 & x_2 \end{bmatrix} \begin{bmatrix} 2x_1 + 2x_2 \\ kx_1 + 4x_2 \end{bmatrix}$

Performing the multiplication:
$x^TAx = 2x_1^2 + 2x_1x_2 + kx_1x_2 + 4x_2^2$

We want this expression to be equal to $2x_1^2 + 4x_1x_2 + 4x_2^2$.
Therefore, we need to find the value of $k$ such that the coefficients
in front of corresponding terms match.

Comparing coefficients: 1. Coefficient of $x_1^2$: $2$ (desired)
$=$ $2$ (current) 2. Coefficient of $x_1x_2$: $4$ (desired)
$=$ $2 + k$ (current) 3. Coefficient of $x_2^2$: $4$ (desired)
$=$ $4$ (current)

From the second equation, $4 = 2 + k$, we find that $k = 2$.

Therefore, for $k = 2$, the associated quadratic form equals
$2x_1^2 + 4x_1x_2 + 4x_2^2$.

    $\textbf{Question 5}$\\
~\\
What is the length of the vector $v=(1,1,…,1)^T∈R^9$ ?\\
~\\
$\textbf{Answer :}$

    The vector $ v = (1, 1, \ldots, 1)^T $ in $ \mathbb{R}^9 $ is
a column vector with all components equal to 1. The length of a vector
(also known as its magnitude or norm) can be computed using the
Euclidean norm formula:

$ \textbar v\textbar{} = \sqrt{v_1^2 + v_2^2 + \ldots + v_n^2} $

In this case, since all components of $ v $ are 1, the formula
simplifies to:

$ \textbar v\textbar{} = \sqrt{1^2 + 1^2 + \ldots + 1^2} $

Since there are 9 components, the expression becomes:

$ \textbar v\textbar{} = \sqrt{9} = 3 $

So, the length of the vector $ v $ is 3 in $ \mathbb{R}^9 $.

    $\textbf{Question 6}$\\
~\\
Given a language model that has been trained to learn word embeddings,
we have reached the next vector representations:

$"cat"=\begin{bmatrix} 1.00 \\ -0.75 \\ 0.00 \\ 0.50 \end{bmatrix}$\\
$"tiger"=\begin{bmatrix} 0.85 \\ -0.65 \\ 0.30 \\ 0.25 \end{bmatrix}$\\
$"lion"=\begin{bmatrix} 0 \\ 0.40 \\ 0.28 \\ 0.67 \end{bmatrix}$\\
$"dog"=\begin{bmatrix} 0.78 \\ -0.32 \\ 0.47 \\ 0.34 \end{bmatrix}$.\\
~\\
Which of the following is correct?\\
a. The largest angle between any pair of distinct vectors is that
between the ``dog'' and the ``lion''.\\
b. The smallest angle between any pair of distinct vectors is that
between the ``cat'' and the ``tiger''.\\
c.~The smallest Euclidean distance between any pair of distinct vectors
is that between the ``cat'' and the ``tiger''.\\
d.~The largest Euclidean distance between any pair of distinct vectors
is that between the ``cat'' and the ``dog''.\\
e. None of these.\\
~\\
$\textbf{Answer :}$

    1.$ \textbf{Angle between "dog" and "lion":}$ $ \cos(\theta) =
\frac{{\mathbf{v}_{\text{dog}} \cdot \mathbf{v}_{\text{lion}}}}{{\|\mathbf{v}_{\text{dog}}\| \|\mathbf{v}_{\text{lion}}\|}}
$ $ \cos(\theta) =
\frac{{0.78 \times 0 + (-0.32) \times 0.40 + 0.47 \times 0.28 + 0.34 \times 0.67}}{{\sqrt{0.78^2 + (-0.32)^2 + 0.47^2 + 0.34^2} \times \sqrt{0^2 + 0.40^2 + 0.28^2 + 0.67^2}}}
$ $ \cos(\theta) \approx 0.53 $ $
\theta \approx \cos^\{-1\}(0.53) \approx 58.2^\circ $

\begin{enumerate}
\def\labelenumi{\arabic{enumi}.}
\setcounter{enumi}{1}
\item
  $\textbf{Angle between "cat" and "tiger":}$ $ \cos(\theta) =
  \frac{{\mathbf{v}_{\text{cat}} \cdot \mathbf{v}_{\text{tiger}}}}{{\|\mathbf{v}_{\text{cat}}\| \|\mathbf{v}_{\text{tiger}}\|}}
  $ $ \cos(\theta) =
  \frac{{1.00 \times 0.85 + (-0.75) \times (-0.65) + 0.00 \times 0.30 + 0.50 \times 0.25}}{{\sqrt{1.00^2 + (-0.75)^2 + 0.00^2 + 0.50^2} \times \sqrt{0.85^2 + (-0.65)^2 + 0.30^2 + 0.25^2}}}
  $ $ \cos(\theta) \approx 0.97 $ $
  \theta \approx \cos^\{-1\}(0.97) \approx 14.0^\circ $
\item
  $\textbf{Euclidean distance between "cat" and "tiger":}$ $
  \text{Distance} =
  \sqrt{\sum_{i=1}^{n} (\mathbf{v}_{\text{cat}}[i] - \mathbf{v}_{\text{tiger}}[i])^2}
  $ $ \text{Distance} =
  \sqrt{(1.00 - 0.85)^2 + (-0.75 - (-0.65))^2 + (0.00 - 0.30)^2 + (0.50 - 0.25)^2}
  $ $ \text{Distance} \approx 0.61 $
\end{enumerate}

4.$ \textbf{Euclidean distance between "cat" and "dog":}$ $
\text{Distance} =
\sqrt{\sum_{i=1}^{n} (\mathbf{v}_{\text{cat}}[i] - \mathbf{v}_{\text{dog}}[i])^2}
$ $ \text{Distance} =
\sqrt{(1.00 - 0.78)^2 + (-0.75 - (-0.32))^2 + (0.00 - 0.47)^2 + (0.50 - 0.34)^2}
$ $ \text{Distance} \approx 0.77 $

Now, let's analyze the options:

\begin{enumerate}
\def\labelenumi{\alph{enumi}.}
\item
  The largest angle is between ``dog'' and ``lion'' (58.2°).
\item
  The smallest angle is between ``cat'' and ``tiger'' (14.0°).
\item
  The smallest Euclidean distance is between ``cat'' and ``tiger''
  (0.61).
\item
  The largest Euclidean distance is between ``cat'' and ``dog'' (0.77).
\end{enumerate}

Based on the calculations, the correct answer is:

$\textbf{b. The smallest angle between any pair of distinct vectors is that between the "cat" and the "tiger".}$

    $\textbf{Question 7}$\\
~\\
Given the vectors $x=\begin{bmatrix} 1 \\ 2 \end{bmatrix}$ ,
$y=\begin{bmatrix} 0 \\ 1 \end{bmatrix}$, Choose the matrix A that
properly defines an inner product and induces the greater distance
between vectors x and y . 1.
$A=\begin{bmatrix} 1 & 0 \\ 0 & 1 \end{bmatrix}$ 2.
$A=\begin{bmatrix} 3 & 0 \\ 0 & 3 \end{bmatrix}$ 3.
$A=\begin{bmatrix} 3 & 1 \\ 1 & 2 \end{bmatrix}$ 4.
$A=\begin{bmatrix} 3 & -1 \\ -1 & 2 \end{bmatrix}$\\
~\\
$\textbf{Answer :}$

    To calculate the distance between vectors $x$ and $y$ with the inner
product induced by a matrix $A$, you can use the formula:

$ d(x, y) = \sqrt{(x - y)^T A (x - y)} $

Now, let's calculate this distance for each given matrix $A$ and
choose the one that results in the greater distance.

\begin{enumerate}
\def\labelenumi{\arabic{enumi}.}
\item
  For $A = \begin{bmatrix} 1 & 0 \\ 0 & 1 \end{bmatrix}$: $ d(x, y) =
  \sqrt{\begin{bmatrix} 1 \\ 1 \end{bmatrix}^T \begin{bmatrix} 1 & 0 \\ 0 & 1 \end{bmatrix} \begin{bmatrix} 1 \\ 1 \end{bmatrix}}
  =
  \sqrt{\begin{bmatrix} 1 & 1 \end{bmatrix} \begin{bmatrix} 1 \\ 1 \end{bmatrix}}
  = \sqrt{2} $
\item
  For $A = \begin{bmatrix} 3 & 0 \\ 0 & 3 \end{bmatrix}$: $ d(x, y) =
  \sqrt{\begin{bmatrix} 1 \\ 1 \end{bmatrix}^T \begin{bmatrix} 3 & 0 \\ 0 & 3 \end{bmatrix} \begin{bmatrix} 1 \\ 1 \end{bmatrix}}
  =
  \sqrt{\begin{bmatrix} 3 & 3 \end{bmatrix} \begin{bmatrix} 1 \\ 1 \end{bmatrix}}
  = \sqrt{6} $
\item
  For $A = \begin{bmatrix} 3 & 1 \\ 1 & 2 \end{bmatrix}$: $ d(x, y) =
  \sqrt{\begin{bmatrix} 1 \\ 1 \end{bmatrix}^T \begin{bmatrix} 3 & 1 \\ 1 & 2 \end{bmatrix} \begin{bmatrix} 1 \\ 1 \end{bmatrix}}
  =
  \sqrt{\begin{bmatrix} 4 & 3 \end{bmatrix} \begin{bmatrix} 1 \\ 1 \end{bmatrix}}
  = \sqrt{7} $
\item
  For $A = \begin{bmatrix} 3 & -1 \\ -1 & 2 \end{bmatrix}$: $ d(x, y)
  =
  \sqrt{\begin{bmatrix} 1 \\ 1 \end{bmatrix}^T \begin{bmatrix} 3 & -1 \\ -1 & 2 \end{bmatrix} \begin{bmatrix} 1 \\ 1 \end{bmatrix}}
  =
  \sqrt{\begin{bmatrix} 2 & 1 \end{bmatrix} \begin{bmatrix} 1 \\ 1 \end{bmatrix}}
  = \sqrt{3} $
\end{enumerate}

Now, comparing the distances: $ \sqrt{2} \textless{} \sqrt{6}
\textless{} \sqrt{7} \textless{} \sqrt{3} $

So, the matrix $A$ that induces the greater distance between vectors
$x$ and $y$ is:
$\textbf{3. $A = \begin{bmatrix} 3 & 1 \\ 1 & 2 \end{bmatrix}$}$

    $\textbf{Question 8}$\\
~\\
What is The angle between the vectors (1,0) and (0,1) ?\\
~\\
$\textbf{Answer :}$

    The angle between the vectors $(1,0)$ and $(0,1)$ can be found using
the dot product formula and the magnitude formula. Let
$\mathbf{A} = (1, 0)$ and $\mathbf{B} = (0, 1)$.

The dot product of two vectors $\mathbf{A} = (a_1, a_2)$ and
$\mathbf{B} = (b_1, b_2)$ is given by: $ \mathbf{A} \cdot \mathbf{B}
= a\_1 \cdot b\_1 + a\_2 \cdot b\_2. $

The magnitude of a vector $\mathbf{V} = (v_1, v_2)$ is given by: $
\textbar{}\mathbf{V}\textbar{} = \sqrt{v_1^2 + v_2^2}. $

Now, let's find the dot product of $(1, 0)$ and $(0, 1)$: $ (1, 0)
\cdot (0, 1) = 1 \cdot 0 + 0 \cdot 1 = 0. $

The magnitudes of the two vectors are: $ \textbar{}\mathbf{A}\textbar{}
= \sqrt{1^2 + 0^2} = 1, \quad \textbar{}\mathbf{B}\textbar{} =
\sqrt{0^2 + 1^2} = 1. $

The formula for the cosine of the angle ($\theta$) between two vectors
$\mathbf{A}$ and $\mathbf{B}$ is given by: $ \cos(\theta) =
\frac{\mathbf{A} \cdot \mathbf{B}}{|\mathbf{A}| \cdot |\mathbf{B}|}. $

Now, plug in the values: $ \cos(\theta) = \frac{0}{1 \cdot 1} = 0. $

To find the angle $\theta$, you can use the inverse cosine function
(arccos): $ \theta = \arccos(0) = \frac{\pi}{2} \text{ radians}. $

So, the angle between the vectors $(1, 0)$ and $(0, 1)$ is
$\frac{\pi}{2}$ radians or $90^\circ$.

    $\textbf{Question 9}$\\
~\\
For an inner product defined by $<x,y>=x^TAy$ , where
$A=\begin{bmatrix} 3 & 6 \\ 6 & 20 \end{bmatrix}$ and $x,y∈R^2$,
what is the approximate angle (in degrees) between the vectors
$a=(1,1)^T$ and $b=(1,−1)^T$ ?\\
~\\
$\textbf{Answer :}$

    For an inner product defined by $<x,y>=x^TAy$, where
$A=\begin{bmatrix} 3 & 6 \\ 6 & 20 \end{bmatrix}$ and
$x,y \in \mathbb{R}^2$, the vectors are: $ a =
\begin{bmatrix} 1 \\ 1 \end{bmatrix}
, \quad b =
\begin{bmatrix} 1 \\ -1 \end{bmatrix}
$
The inner product is calculated as: $ \langle a, b \rangle = a^TA b
=
\begin{bmatrix} 1 & 1 \end{bmatrix} \begin{bmatrix} 3 & 6 \\ 6 & 20 \end{bmatrix} \begin{bmatrix} 1 \\ -1 \end{bmatrix}
$
$ =
\begin{bmatrix} 1 & 1 \end{bmatrix} \begin{bmatrix} -3 \\ -14 \end{bmatrix}
= -17 $
The magnitudes are: $||a|| = \sqrt{2}$, $||b|| = \sqrt{2}$.
The cosine of the angle $\theta$ is: $ \cos(\theta) =
\frac{\langle a, b \rangle}{||a|| \cdot ||b||} =
\frac{-17}{\sqrt{2} \cdot \sqrt{2}} = \frac{-17}{2} $

The angle $\theta$ is found using the arccosine function: $ \theta =
\cos^\{-1\}\left(\frac{-17}{2}\right) \approx 118.6^\circ $

So, the approximate angle between the vectors $a$ and $b$ is
$118.6^\circ$.

    $\textbf{Question 10}$\\
~\\
Let $a=(1,−1,1)^T$ , $b=(1,−1,−1)^T$ and $θ$ the angle between the
two vectors. Find the cosθ\\
$\textbf{Answer :}$

    Let $\mathbf{a} = \begin{bmatrix} 1 \\ -1 \\ 1 \end{bmatrix}$ and
$\mathbf{b} = \begin{bmatrix} 1 \\ -1 \\ -1 \end{bmatrix}$ be two
vectors. The angle between them, denoted by $\theta$, can be found
using the formula:

$ \cos(\theta) =
\frac{\mathbf{a} \cdot \mathbf{b}}{\|\mathbf{a}\| \|\mathbf{b}\|} $
where $\cdot$ represents the dot product and $\|\mathbf{v}\|$
represents the magnitude of vector $\mathbf{v}$.

The dot product $\mathbf{a} \cdot \mathbf{b}$ is calculated as:

$ \mathbf{a} \cdot \mathbf{b} = (1)(1) + (-1)(-1) + (1)(-1) = 1 + 1 - 1
= 1 $ The magnitudes are given by:

$ \textbar{}\mathbf{a}\textbar{} = \sqrt{1^2 + (-1)^2 + 1^2} = \sqrt{3}
$ $ \textbar{}\mathbf{b}\textbar{} = \sqrt{1^2 + (-1)^2 + (-1)^2} =
\sqrt{3} $ Now, substitute these values into the formula for the cosine
of the angle:

$ \cos(\theta) =
\frac{\mathbf{a} \cdot \mathbf{b}}{\|\mathbf{a}\| \|\mathbf{b}\|} =
\frac{1}{\sqrt{3} \cdot \sqrt{3}} = \frac{1}{3} $ Therefore, the cosine
of the angle $\theta$ between vectors $\mathbf{a}$ and
$\mathbf{b}$ is $\frac{1}{3}$.

    $\textbf{Question 11}$\\
~\\
Which of the following vectors is or can be orthogonal to\\
$x=\begin{bmatrix} 1 \\ k \\ 0 \\-2 \end{bmatrix}$ for k\textgreater0
?\\
$a=\begin{bmatrix} 1 \\ 2 \\ 3 \\-4 \end{bmatrix}$\\
$b=\begin{bmatrix} 0 \\ 0 \\ 0 \\ 1/4 \end{bmatrix}$\\
$c=\begin{bmatrix} 1 \\ 0 \\ 5 \\ 1/2 \end{bmatrix}$\\
$d=\begin{bmatrix} 1 \\ 1 \\ 1 \\ -1 \end{bmatrix}$

$\textbf{Answer :}$
    For vector $\mathbf{a}$: $ \mathbf{x} \cdot \mathbf{a} =
\begin{bmatrix} 1 \\ k \\ 0 \\ -2 \end{bmatrix}
\cdot 
\begin{bmatrix} 1 \\ 2 \\ 3 \\ -4 \end{bmatrix}
= (1)(1) + (k)(2) + (0)(3) + (-2)(-4) = 1 + 2k + 8 = 9 + 2k $ For
$\mathbf{x}$ and $\mathbf{a}$ to be orthogonal, $9 + 2k = 0$.
However, there is no value of $k$ that satisfies this equation, so
$\mathbf{a}$ is not orthogonal to $\mathbf{x}$.

For vector $\mathbf{b}$: $ \mathbf{x} \cdot \mathbf{b} =
\begin{bmatrix} 1 \\ k \\ 0 \\ -2 \end{bmatrix}
\cdot 
\begin{bmatrix} 0 \\ 0 \\ 0 \\ \frac{1}{4} \end{bmatrix}
= (1)(0) + (k)(0) + (0)(0) + (-2)\left(\frac{1}{4}\right) = -\frac{1}{2}
$ For $\mathbf{x}$ and $\mathbf{b}$ to be orthogonal,
$-\frac{1}{2} = 0$, which is not possible. Therefore, $\mathbf{b}$
is not orthogonal to $\mathbf{x}$.

For vector $\mathbf{c}$: $ \mathbf{x} \cdot \mathbf{c} =
\begin{bmatrix} 1 \\ k \\ 0 \\ -2 \end{bmatrix}
\cdot 
\begin{bmatrix} 1 \\ 0 \\ 5 \\ \frac{1}{2} \end{bmatrix}
= (1)(1) + (k)(0) + (0)(5) + (-2)\left(\frac{1}{2}\right) = 1 - 1 = 0 $
For $\mathbf{x}$ and $\mathbf{c}$ to be orthogonal, $0 = 0$, which
is true. So, $\mathbf{c}$ is orthogonal to $\mathbf{x}$.
For vector $\mathbf{d}$: $ \mathbf{x} \cdot \mathbf{d} =
\begin{bmatrix} 1 \\ k \\ 0 \\ -2 \end{bmatrix}
\cdot 
\begin{bmatrix} 1 \\ 1 \\ 1 \\ -1 \end{bmatrix}
= (1)(1) + (k)(1) + (0)(1) + (-2)(-1) = 1 + k + 2 = 3 + k $ For
$\mathbf{x}$ and $\mathbf{d}$ to be orthogonal, $3 + k = 0$, which
is satisfied when $k = -3$. However, the given condition is $k > 0$,
so $\mathbf{d}$ is not orthogonal to $\mathbf{x}$ for $k > 0$.

In conclusion, only vector $\mathbf{c}$ is orthogonal to
$\mathbf{x}$ for $k > 0$.

    $\textbf{Question 12}$\\
~\\
Which of the following vectors is orthogonal to $x=(1,1,2)^T$\\
a. $a=(-1,1,1)^T$\\
b. $b=(2,-2,0)^T$\\
c.~$c=(0,1,1)^T$\\
d.~$d=(0,2,0)^T$\\
~\\
$\textbf{Answer :}$

    Two vectors are orthogonal if their dot product is zero. The dot product
of two vectors $a = (a_1, a_2, a_3)^T$ and $b = (b_1, b_2, b_3)^T$
is given by:

$ a \cdot b = a\_1b\_1 + a\_2b\_2 + a\_3b\_3 $

Let's calculate the dot product for each given vector with
$x = (1, 1, 2)^T$:

\begin{enumerate}
\def\labelenumi{\alph{enumi}.}
\item
  $a = (-1, 1, 1)^T$: $ x \cdot a = (1)(-1) + (1)(1) + (2)(1) = -1 +
  1 + 2 = 2 $ The dot product is not zero, so vector $a$ is not
  orthogonal to $x$.
\item
  $b = (2, -2, 0)^T$: $ x \cdot b = (1)(2) + (1)(-2) + (2)(0) = 2 - 2
  + 0 = 0 $ The dot product is zero, so vector $b$ is orthogonal to
  $x$.
\item
  $c = (0, 1, 1)^T$: $ x \cdot c = (1)(0) + (1)(1) + (2)(1) = 0 + 1 +
  2 = 3 $ The dot product is not zero, so vector $c$ is not
  orthogonal to $x$.
\item
  $d = (0, 2, 0)^T$: $ x \cdot d = (1)(0) + (1)(2) + (2)(0) = 0 + 2 +
  0 = 2 $ The dot product is not zero, so vector $d$ is not
  orthogonal to $x$.
\end{enumerate}

Therefore, the only vector that is orthogonal to $x$ among the given
options is $b = (2, -2, 0)^T$, so the correct answer is b.

    $\textbf{Question 12}$\\
~\\
Which of the following matrices is/are orthogonal ?\\
$A=\begin{bmatrix} 1 & 0 & 0 \\ 0 & 1/\sqrt{2} & 1/\sqrt{2} \\ 0 & -1/\sqrt{2} & 1/\sqrt{2} \end{bmatrix}$\\
$B=\begin{bmatrix} 1 & 0 & -1 \\ 0 & 1 & 0 \\ 1 & 0 & 1 \end{bmatrix}$\\
$C=\begin{bmatrix} 0 & 0 & 0 \\ 1 & 2 & 0 \\ 2 & 1 & 0 \end{bmatrix}$\\
~\\
$\textbf{Answer :}$

    \begin{enumerate}
\def\labelenumi{\arabic{enumi}.}
\item
  Matrix $A$: $ A^T \cdot A =
  \begin{bmatrix} 1 & 0 & 0 \\ 0 & \frac{1}{\sqrt{2}} & -\frac{1}{\sqrt{2}} \\ 0 & \frac{1}{\sqrt{2}} & \frac{1}{\sqrt{2}} \end{bmatrix}
  \cdot 
  \begin{bmatrix} 1 & 0 & 0 \\ 0 & \frac{1}{\sqrt{2}} & \frac{1}{\sqrt{2}} \\ 0 & -\frac{1}{\sqrt{2}} & \frac{1}{\sqrt{2}} \end{bmatrix}
  =
  \begin{bmatrix} 1 & 0 & 0 \\ 0 & 1 & 0 \\ 0 & 0 & 1 \end{bmatrix}
  $
\item
  Matrix $B$: $ B^T \cdot B =
  \begin{bmatrix} 1 & 0 & 1 \\ 0 & 1 & 0 \\ -1 & 0 & 1 \end{bmatrix}
  \cdot 
  \begin{bmatrix} 1 & 0 & -1 \\ 0 & 1 & 0 \\ 1 & 0 & 1 \end{bmatrix}
  =
  \begin{bmatrix} 3 & 0 & 0 \\ 0 & 2 & 0 \\ 0 & 0 & 3 \end{bmatrix}
  $
\item
  Matrix $C$: $ C^T \cdot C =
  \begin{bmatrix} 0 & 1 & 2 \\ 0 & 0 & 1 \\ 0 & 0 & 0 \end{bmatrix}
  \cdot 
  \begin{bmatrix} 0 & 0 & 0 \\ 1 & 2 & 0 \\ 2 & 1 & 0 \end{bmatrix}
  =
  \begin{bmatrix} 5 & 2 & 0 \\ 2 & 5 & 0 \\ 0 & 0 & 0 \end{bmatrix}
  $
\end{enumerate}

Therefore, only matrix $A$ is orthogonal.

    $\textbf{Question 14}$\\
~\\
For which value of k the following set of vectors\\
$x1=\begin{bmatrix} 1 \\ 1 \\ 1 \end{bmatrix}$,\\
$x2=\begin{bmatrix} 1 \\ 2k \\ 1 \end{bmatrix}$,\\
$x1=\begin{bmatrix} 1 \\ 0 \\ -1 \end{bmatrix}$,\\
forms an orthogonal basis in $R^3$ ?\\
~\\
$\textbf{Answer :}$

    To check if the set of vectors forms an orthogonal basis in
$\mathbb{R}^3$, we need to ensure that each vector in the set is
orthogonal to every other vector. Additionally, each vector in the set
must be non-zero.

Let's denote the vectors as $x_1, x_2, x_3$ for simplicity:

$ x\_1 =
\begin{bmatrix} 1 \\ 1 \\ 1 \end{bmatrix}
, $ $ x\_2 =
\begin{bmatrix} 1 \\ 2k \\ 1 \end{bmatrix}
, $ $ x\_3 =
\begin{bmatrix} 1 \\ 0 \\ -1 \end{bmatrix}
. $
Now, for these vectors to form an orthogonal basis, the dot product of
every pair of distinct vectors should be zero. The dot product of two
vectors $u$ and $v$ is given by $u \cdot v = u^Tv$. Therefore, for
$x_i \neq x_j$, $i \neq j$, the dot product $x_i^Tx_j$ should be
zero.

Let's check the conditions:

\begin{enumerate}
\def\labelenumi{\arabic{enumi}.}
\tightlist
\item
  $x_1^Tx_2 = 0$: $
  \begin{bmatrix} 1 & 1 & 1 \end{bmatrix} \begin{bmatrix} 1 \\ 2k \\ 1 \end{bmatrix}
  = 1 + 2k + 1 = 2k + 2 $
\end{enumerate}

For $x_1^Tx_2$ to be zero, $k$ must be equal to $-1$.

\begin{enumerate}
\def\labelenumi{\arabic{enumi}.}
\setcounter{enumi}{1}
\item
  $x_1^Tx_3 = 0$: $
  \begin{bmatrix} 1 & 1 & 1 \end{bmatrix} \begin{bmatrix} 1 \\ 0 \\ -1 \end{bmatrix}
  = 1 + 0 - 1 = 0 $
\item
  $x_2^Tx_3 = 0$: $
  \begin{bmatrix} 1 & 2k & 1 \end{bmatrix} \begin{bmatrix} 1 \\ 0 \\ -1 \end{bmatrix}
  = 1 + 0 - 1 = 0 $
\end{enumerate}

So, $k = -1$ is the value for which the set of vectors forms an
orthogonal basis in $\mathbb{R}^3$.

    $\textbf{Question 15}$\\
~\\
Which of the following sets of vectors forms an orthonormal basis?\\
A. $ x\_1 = \frac{1}{\sqrt{2}}
\begin{bmatrix} 1 \\ 1 \\ 1 \end{bmatrix}
$, $ x\_2 = \frac{1}{\sqrt{6}}
\begin{bmatrix} -2 \\ 1 \\ 1 \end{bmatrix}
$, $ x\_3 = \frac{1}{2\sqrt{3}}
\begin{bmatrix} 0 \\ -3 \\ 3 \end{bmatrix}
$\\

\begin{enumerate}
\def\labelenumi{\Alph{enumi}.}
\setcounter{enumi}{1}
\item
  $ x\_1 = \frac{1}{3}
  \begin{bmatrix} 2 \\ 1 \\ 2 \end{bmatrix}
  $, $ x\_2 = \frac{1}{3}
  \begin{bmatrix} 2 \\ 2 \\ 1 \end{bmatrix}
  $, $ x\_3 = \frac{1}{\sqrt{10}}
  \begin{bmatrix} 0 \\ 3 \\ 1 \end{bmatrix}
  $\\
\item
  $ x\_1 = \frac{1}{3}
  \begin{bmatrix} 1 \\ 1 \\ 1 \end{bmatrix}
  $, $ x\_2 = \frac{1}{6}
  \begin{bmatrix} 2 \\ -1 \\ -1 \end{bmatrix}
  $, $ x\_3 = \frac{1}{2}
  \begin{bmatrix} 0 \\ 1 \\ -1 \end{bmatrix}
  $\\
\item
  $ x\_1 = \frac{1}{\sqrt{2}}
  \begin{bmatrix} 1 \\ 0 \\ 1 \end{bmatrix}
  $, $ x\_2 = \frac{1}{\sqrt{2}}
  \begin{bmatrix} -1 \\ -1 \\ 0 \end{bmatrix}
  $, $ x\_3 = \frac{1}{\sqrt{2}}
  \begin{bmatrix} 0 \\ 1 \\ 1 \end{bmatrix}
  $\\
  $\textbf{Answer :}$
\end{enumerate}

    $\textbf{Set A:}$
The vectors in set A are:\\
$x_1 = \frac{1}{\sqrt{2}} \begin{bmatrix} 1 \\ 1 \\ 1 \end{bmatrix} \\ x_2 = \frac{1}{\sqrt{6}} \begin{bmatrix} -2 \\ 1 \\ 1 \end{bmatrix} \\ x_3 = \frac{1}{2\sqrt{3}} \begin{bmatrix} 0 \\ -3 \\ 3 \end{bmatrix} $

$\textbf{Dot Products:}$\\
$ x_1 \cdot x_2 = \frac{1}{\sqrt{2}} \begin{bmatrix} 1 \\ 1 \\ 1 \end{bmatrix} \cdot \frac{1}{\sqrt{6}} \begin{bmatrix} -2 \\ 1 \\ 1 \end{bmatrix} = 0 \\ x_1 \cdot x_3 = \frac{1}{\sqrt{2}} \begin{bmatrix} 1 \\ 1 \\ 1 \end{bmatrix} \cdot \frac{1}{2\sqrt{3}} \begin{bmatrix} 0 \\ -3 \\ 3 \end{bmatrix} = 0 \\ x_2 \cdot x_3 = \frac{1}{\sqrt{6}} \begin{bmatrix} -2 \\ 1 \\ 1 \end{bmatrix} \cdot \frac{1}{2\sqrt{3}} \begin{bmatrix} 0 \\ -3 \\ 3 \end{bmatrix} = 0 $

$\textbf{Magnitudes:}$\\
$ \|x_1\| = \left\| \frac{1}{\sqrt{2}} \begin{bmatrix} 1 \\ 1 \\ 1 \end{bmatrix} \right\| = 1 \\ \|x_2\| = \left\| \frac{1}{\sqrt{6}} \begin{bmatrix} -2 \\ 1 \\ 1 \end{bmatrix} \right\| = 1 \\ \|x_3\| = \left\| \frac{1}{2\sqrt{3}} \begin{bmatrix} 0 \\ -3 \\ 3 \end{bmatrix} \right\| = 1 $

$\textbf{Summary:}$ Set A forms an orthonormal basis.

    $\textbf{Question 16}$\\
~\\
The orthogonal complement in $R^3$ of the span of the vectors\\
$v1=\begin{bmatrix} 1 \\ 1 \\ 1 \end{bmatrix}$,
$v2=\begin{bmatrix} 1 \\ 2 \\ 0 \end{bmatrix}$,
$v1=\begin{bmatrix} 2 \\ 3 \\ 1 \end{bmatrix}$\\
is :\\
a. a plane\\
b. the empty set\\
c.~a line\\
d.~the whole 3-dimensional space\\
e. a point\\
~\\
$\textbf{Answer :}$

    To find the orthogonal complement of the span of the given vectors in $
\mathbb{R}^3 $, we can use the following steps:

\begin{enumerate}
\def\labelenumi{\arabic{enumi}.}
\tightlist
\item
  Create a matrix $ A $ whose columns are the given vectors.
\item
  Find the reduced row-echelon form (RREF) of $ A $.
\item
  Identify the pivot columns of $ A $.
\item
  The orthogonal complement is the span of the vectors corresponding to
  the non-pivot columns of $ A $.
\end{enumerate}

The matrix $ A $ is formed by the given vectors:

$ A =
\begin{bmatrix} 1 & 1 & 2 \\ 1 & 2 & 3 \\ 1 & 0 & 1 \end{bmatrix}
$

Now, find the reduced row-echelon form (RREF) of $ A $:
$ \text{RREF}(A) =
\begin{bmatrix} 1 & 0 & 1 \\ 0 & 1 & 1 \\ 0 & 0 & 0 \end{bmatrix}
$

The pivot columns are the first and second columns. The third column is
a non-pivot column.

So, the orthogonal complement is the span of the vector corresponding to
the non-pivot column:

$ \text{Orthogonal Complement} = \text{span}\left\{
\begin{bmatrix} 1 \\ 1 \\ 0 \end{bmatrix}
\right\} $

This is a line in $ \mathbb{R}^3 $. Therefore, the correct answer
is:
\begin{enumerate}
\def\labelenumi{\alph{enumi}.}
\setcounter{enumi}{2}
\tightlist
\item
  a line
\end{enumerate}

    $\textbf{Question 17}$\\
The orthogonal projection of $x=(1,1,1)^T$ on the vector
$b=(\sqrt{2},−1,1)^T$ is equal to:\\
~\\
$\textbf{Answer :}$

    The orthogonal projection of $ \mathbf{x} = (1,1,1)^T $ onto $
\mathbf{b} = (\sqrt{2}, -1, 1)^T $ is given by:

$ \text{proj}\_\mathbf{b}(\mathbf{x}) =
\frac{\mathbf{x} \cdot \mathbf{b}}{\|\mathbf{b}\|^2} \cdot \mathbf{b} $

Calculating the dot product and the norm:

$
\mathbf{x} \cdot \mathbf{b} = \sqrt{2} - 1 + 1 = \sqrt{2} \\
\|\mathbf{b}\| = \sqrt{2 + 1 + 1} = \sqrt{4} = 2
$

Substituting these values into the projection formula:

$ \text{proj}\_\mathbf{b}(\mathbf{x}) = \frac{\sqrt{2}}{2^2}
\cdot (\sqrt{2}, -1, 1) $

Simplifying:

$ \text{proj}\_\mathbf{b}(\mathbf{x}) = \left(\frac{1}{2},
-\frac{\sqrt{2}}{4}, \frac{\sqrt{2}}{4}\right) $

Therefore, the orthogonal projection of $ \mathbf{x} = (1,1,1)^T $
onto $ \mathbf{b} = (\sqrt{2}, -1, 1)^T $ is $ \left(\frac{1}{2},
-\frac{\sqrt{2}}{4}, \frac{\sqrt{2}}{4}\right) $.

    $\textbf{Question 18}$\\
Calculate The projection matrix $P_π$ onto the line through the origin
spanned by $b=(1,2,1)^T$\\
~\\
$\textbf{Answer :}$

    To calculate the projection matrix $ P\_\{\pi\} $ onto the line
through the origin spanned by $ b = (1, 2, 1)^T $, we use the
formula: $ P\_\{\pi\} = \frac{bb^T}{b^Tb} $

\begin{enumerate}
\def\labelenumi{\arabic{enumi}.}
\item
  Transpose of $ b $ ($ b^T $): $ b^T =
  \begin{bmatrix} 1 & 2 & 1 \end{bmatrix}
  $
\item
  Outer product of $ b $ with itself ($ bb^T $): $ bb^T =
  \begin{bmatrix} 1 & 2 & 1 \\ 2 & 4 & 2 \\ 1 & 2 & 1 \end{bmatrix}
  $
\item
  Dot product of $ b $ with itself ($ b^Tb $): $ b^Tb = 6 $
\item
  Projection matrix $ P\_\{\pi\} $: $ P\_\{\pi\} = \frac{1}{6}
  \begin{bmatrix} 1 & 2 & 1 \\ 2 & 4 & 2 \\ 1 & 2 & 1 \end{bmatrix}
  $
\end{enumerate}

So, the projection matrix $ P\_\{\pi\} $ is: $ P\_\{\pi\} =
\frac{1}{6}
\begin{bmatrix} 1 & 2 & 1 \\ 2 & 4 & 2 \\ 1 & 2 & 1 \end{bmatrix}
$

    $\textbf{Question 19}$\\
After rotating the two vectors $x1=(1,1)^T$ and $y1=(1,−2)^T$ by 50
degrees clockwise around the origin, what the distance between the two
new vectors ?\\
~\\
$\textbf{Answer :}$

    Given vectors $x_1 = \begin{bmatrix}1 \\ 1\end{bmatrix}$ and
$y_1 = \begin{bmatrix}1 \\ -2\end{bmatrix}$, and the rotation angle
$-50$ degrees, the rotated vectors $x_2$ and $y_2$ are obtained as
follows:

$ R(-50) =
\begin{bmatrix}
0.64279 & 0.76604 \\
-0.76604 & 0.64279
\end{bmatrix}
$

$ x\_2 = R(-50) \cdot x\_1 =
\begin{bmatrix}
0.64279 & 0.76604 \\
-0.76604 & 0.64279
\end{bmatrix}
\begin{bmatrix}
1 \\
1
\end{bmatrix}
$

$ y\_2 = R(-50) \cdot y\_1 =
\begin{bmatrix}
0.64279 & 0.76604 \\
-0.76604 & 0.64279
\end{bmatrix}
\begin{bmatrix}
1 \\
-2
\end{bmatrix}
$

The Euclidean distance between these vectors is given by:

$ \text{Distance} =
\sqrt{(x_{2_1} - y_{2_1})^2 + (x_{2_2} - y_{2_2})^2} $

Substituting the values and calculating:

$ \text{Distance} =
\sqrt{(\text{result of } x_{2_1} - y_{2_1})^2 + (\text{result of } x_{2_2} - y_{2_2})^2}
$

    $\textbf{Question 20}$\\
If we rotate the vector $C=\begin{bmatrix} 2 \\ -2 \end{bmatrix}$ by
$45∘$, we obtain the vector\\
~\\
$\textbf{Answer :}$

    To rotate a vector in two dimensions counterclockwise by an angle
$\theta$, you can use the following rotation matrix:

$ R =
\begin{bmatrix} \cos(\theta) & -\sin(\theta) \\ \sin(\theta) & \cos(\theta) \end{bmatrix}
$

In this case, you want to rotate the vector
$C = \begin{bmatrix} 2 \\ -2 \end{bmatrix}$ by $45^\circ$. The angle
in radians for $45^\circ$ is $\frac{\pi}{4}$. Therefore, the
rotation matrix would be:

$ R =
\begin{bmatrix} \cos\left(\frac{\pi}{4}\right) & -\sin\left(\frac{\pi}{4}\right) \\ \sin\left(\frac{\pi}{4}\right) & \cos\left(\frac{\pi}{4}\right) \end{bmatrix}
$

Now, you can multiply this rotation matrix by the vector $C$ to get
the rotated vector $C'$:

$ C' = R \cdot C $

Calculating this, we get:

$ C' =
\begin{bmatrix} \sqrt{2}/2 & -\sqrt{2}/2 \\ \sqrt{2}/2 & \sqrt{2}/2 \end{bmatrix}
\cdot \begin{bmatrix} 2 \\ -2 \end{bmatrix}
$

Performing the matrix multiplication, we get:

$ C' =
\begin{bmatrix} (\sqrt{2}/2) \cdot 2 + (-\sqrt{2}/2) \cdot (-2) \\ (\sqrt{2}/2) \cdot 2 + (\sqrt{2}/2) \cdot (-2) \end{bmatrix}
$

Simplify the expressions:

$ C' =
\begin{bmatrix} \sqrt{2} + \sqrt{2} \\ \sqrt{2} - \sqrt{2} \end{bmatrix}
$

Finally:

$ C' =
\begin{bmatrix} 2\sqrt{2} \\ 0 \end{bmatrix}
$

So, the vector obtained by rotating
$C = \begin{bmatrix} 2 \\ -2 \end{bmatrix}$ by $45^\circ$ is
$C' = \begin{bmatrix} 2\sqrt{2} \\ 0 \end{bmatrix}$.

    \begin{tcolorbox}[breakable, size=fbox, boxrule=1pt, pad at break*=1mm,colback=cellbackground, colframe=cellborder]
\prompt{In}{incolor}{ }{\boxspacing}
\begin{Verbatim}[commandchars=\\\{\}]

\end{Verbatim}
\end{tcolorbox}


    % Add a bibliography block to the postdoc
    
    
    
\end{document}
